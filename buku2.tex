% Options for packages loaded elsewhere
% Options for packages loaded elsewhere
\PassOptionsToPackage{unicode}{hyperref}
\PassOptionsToPackage{hyphens}{url}
\PassOptionsToPackage{dvipsnames,svgnames,x11names}{xcolor}
%
\documentclass[
  letterpaper,
  DIV=11,
  numbers=noendperiod]{scrartcl}
\usepackage{xcolor}
\usepackage{amsmath,amssymb}
\setcounter{secnumdepth}{-\maxdimen} % remove section numbering
\usepackage{iftex}
\ifPDFTeX
  \usepackage[T1]{fontenc}
  \usepackage[utf8]{inputenc}
  \usepackage{textcomp} % provide euro and other symbols
\else % if luatex or xetex
  \usepackage{unicode-math} % this also loads fontspec
  \defaultfontfeatures{Scale=MatchLowercase}
  \defaultfontfeatures[\rmfamily]{Ligatures=TeX,Scale=1}
\fi
\usepackage{lmodern}
\ifPDFTeX\else
  % xetex/luatex font selection
\fi
% Use upquote if available, for straight quotes in verbatim environments
\IfFileExists{upquote.sty}{\usepackage{upquote}}{}
\IfFileExists{microtype.sty}{% use microtype if available
  \usepackage[]{microtype}
  \UseMicrotypeSet[protrusion]{basicmath} % disable protrusion for tt fonts
}{}
\makeatletter
\@ifundefined{KOMAClassName}{% if non-KOMA class
  \IfFileExists{parskip.sty}{%
    \usepackage{parskip}
  }{% else
    \setlength{\parindent}{0pt}
    \setlength{\parskip}{6pt plus 2pt minus 1pt}}
}{% if KOMA class
  \KOMAoptions{parskip=half}}
\makeatother
% Make \paragraph and \subparagraph free-standing
\makeatletter
\ifx\paragraph\undefined\else
  \let\oldparagraph\paragraph
  \renewcommand{\paragraph}{
    \@ifstar
      \xxxParagraphStar
      \xxxParagraphNoStar
  }
  \newcommand{\xxxParagraphStar}[1]{\oldparagraph*{#1}\mbox{}}
  \newcommand{\xxxParagraphNoStar}[1]{\oldparagraph{#1}\mbox{}}
\fi
\ifx\subparagraph\undefined\else
  \let\oldsubparagraph\subparagraph
  \renewcommand{\subparagraph}{
    \@ifstar
      \xxxSubParagraphStar
      \xxxSubParagraphNoStar
  }
  \newcommand{\xxxSubParagraphStar}[1]{\oldsubparagraph*{#1}\mbox{}}
  \newcommand{\xxxSubParagraphNoStar}[1]{\oldsubparagraph{#1}\mbox{}}
\fi
\makeatother

\usepackage{color}
\usepackage{fancyvrb}
\newcommand{\VerbBar}{|}
\newcommand{\VERB}{\Verb[commandchars=\\\{\}]}
\DefineVerbatimEnvironment{Highlighting}{Verbatim}{commandchars=\\\{\}}
% Add ',fontsize=\small' for more characters per line
\usepackage{framed}
\definecolor{shadecolor}{RGB}{241,243,245}
\newenvironment{Shaded}{\begin{snugshade}}{\end{snugshade}}
\newcommand{\AlertTok}[1]{\textcolor[rgb]{0.68,0.00,0.00}{#1}}
\newcommand{\AnnotationTok}[1]{\textcolor[rgb]{0.37,0.37,0.37}{#1}}
\newcommand{\AttributeTok}[1]{\textcolor[rgb]{0.40,0.45,0.13}{#1}}
\newcommand{\BaseNTok}[1]{\textcolor[rgb]{0.68,0.00,0.00}{#1}}
\newcommand{\BuiltInTok}[1]{\textcolor[rgb]{0.00,0.23,0.31}{#1}}
\newcommand{\CharTok}[1]{\textcolor[rgb]{0.13,0.47,0.30}{#1}}
\newcommand{\CommentTok}[1]{\textcolor[rgb]{0.37,0.37,0.37}{#1}}
\newcommand{\CommentVarTok}[1]{\textcolor[rgb]{0.37,0.37,0.37}{\textit{#1}}}
\newcommand{\ConstantTok}[1]{\textcolor[rgb]{0.56,0.35,0.01}{#1}}
\newcommand{\ControlFlowTok}[1]{\textcolor[rgb]{0.00,0.23,0.31}{\textbf{#1}}}
\newcommand{\DataTypeTok}[1]{\textcolor[rgb]{0.68,0.00,0.00}{#1}}
\newcommand{\DecValTok}[1]{\textcolor[rgb]{0.68,0.00,0.00}{#1}}
\newcommand{\DocumentationTok}[1]{\textcolor[rgb]{0.37,0.37,0.37}{\textit{#1}}}
\newcommand{\ErrorTok}[1]{\textcolor[rgb]{0.68,0.00,0.00}{#1}}
\newcommand{\ExtensionTok}[1]{\textcolor[rgb]{0.00,0.23,0.31}{#1}}
\newcommand{\FloatTok}[1]{\textcolor[rgb]{0.68,0.00,0.00}{#1}}
\newcommand{\FunctionTok}[1]{\textcolor[rgb]{0.28,0.35,0.67}{#1}}
\newcommand{\ImportTok}[1]{\textcolor[rgb]{0.00,0.46,0.62}{#1}}
\newcommand{\InformationTok}[1]{\textcolor[rgb]{0.37,0.37,0.37}{#1}}
\newcommand{\KeywordTok}[1]{\textcolor[rgb]{0.00,0.23,0.31}{\textbf{#1}}}
\newcommand{\NormalTok}[1]{\textcolor[rgb]{0.00,0.23,0.31}{#1}}
\newcommand{\OperatorTok}[1]{\textcolor[rgb]{0.37,0.37,0.37}{#1}}
\newcommand{\OtherTok}[1]{\textcolor[rgb]{0.00,0.23,0.31}{#1}}
\newcommand{\PreprocessorTok}[1]{\textcolor[rgb]{0.68,0.00,0.00}{#1}}
\newcommand{\RegionMarkerTok}[1]{\textcolor[rgb]{0.00,0.23,0.31}{#1}}
\newcommand{\SpecialCharTok}[1]{\textcolor[rgb]{0.37,0.37,0.37}{#1}}
\newcommand{\SpecialStringTok}[1]{\textcolor[rgb]{0.13,0.47,0.30}{#1}}
\newcommand{\StringTok}[1]{\textcolor[rgb]{0.13,0.47,0.30}{#1}}
\newcommand{\VariableTok}[1]{\textcolor[rgb]{0.07,0.07,0.07}{#1}}
\newcommand{\VerbatimStringTok}[1]{\textcolor[rgb]{0.13,0.47,0.30}{#1}}
\newcommand{\WarningTok}[1]{\textcolor[rgb]{0.37,0.37,0.37}{\textit{#1}}}

\usepackage{longtable,booktabs,array}
\usepackage{calc} % for calculating minipage widths
% Correct order of tables after \paragraph or \subparagraph
\usepackage{etoolbox}
\makeatletter
\patchcmd\longtable{\par}{\if@noskipsec\mbox{}\fi\par}{}{}
\makeatother
% Allow footnotes in longtable head/foot
\IfFileExists{footnotehyper.sty}{\usepackage{footnotehyper}}{\usepackage{footnote}}
\makesavenoteenv{longtable}
\usepackage{graphicx}
\makeatletter
\newsavebox\pandoc@box
\newcommand*\pandocbounded[1]{% scales image to fit in text height/width
  \sbox\pandoc@box{#1}%
  \Gscale@div\@tempa{\textheight}{\dimexpr\ht\pandoc@box+\dp\pandoc@box\relax}%
  \Gscale@div\@tempb{\linewidth}{\wd\pandoc@box}%
  \ifdim\@tempb\p@<\@tempa\p@\let\@tempa\@tempb\fi% select the smaller of both
  \ifdim\@tempa\p@<\p@\scalebox{\@tempa}{\usebox\pandoc@box}%
  \else\usebox{\pandoc@box}%
  \fi%
}
% Set default figure placement to htbp
\def\fps@figure{htbp}
\makeatother





\setlength{\emergencystretch}{3em} % prevent overfull lines

\providecommand{\tightlist}{%
  \setlength{\itemsep}{0pt}\setlength{\parskip}{0pt}}



 


\KOMAoption{captions}{tableheading}
\makeatletter
\@ifpackageloaded{caption}{}{\usepackage{caption}}
\AtBeginDocument{%
\ifdefined\contentsname
  \renewcommand*\contentsname{Table of contents}
\else
  \newcommand\contentsname{Table of contents}
\fi
\ifdefined\listfigurename
  \renewcommand*\listfigurename{List of Figures}
\else
  \newcommand\listfigurename{List of Figures}
\fi
\ifdefined\listtablename
  \renewcommand*\listtablename{List of Tables}
\else
  \newcommand\listtablename{List of Tables}
\fi
\ifdefined\figurename
  \renewcommand*\figurename{Figure}
\else
  \newcommand\figurename{Figure}
\fi
\ifdefined\tablename
  \renewcommand*\tablename{Table}
\else
  \newcommand\tablename{Table}
\fi
}
\@ifpackageloaded{float}{}{\usepackage{float}}
\floatstyle{ruled}
\@ifundefined{c@chapter}{\newfloat{codelisting}{h}{lop}}{\newfloat{codelisting}{h}{lop}[chapter]}
\floatname{codelisting}{Listing}
\newcommand*\listoflistings{\listof{codelisting}{List of Listings}}
\makeatother
\makeatletter
\makeatother
\makeatletter
\@ifpackageloaded{caption}{}{\usepackage{caption}}
\@ifpackageloaded{subcaption}{}{\usepackage{subcaption}}
\makeatother
\usepackage{bookmark}
\IfFileExists{xurl.sty}{\usepackage{xurl}}{} % add URL line breaks if available
\urlstyle{same}
\hypersetup{
  colorlinks=true,
  linkcolor={blue},
  filecolor={Maroon},
  citecolor={Blue},
  urlcolor={Blue},
  pdfcreator={LaTeX via pandoc}}


\author{}
\date{}
\begin{document}


\section{Paradigma Triune‑Intelligence Smart Engineering
(TISE)}\label{paradigma-triuneintelligence-smart-engineering-tise}

\textbf{Subjudul:} Orkestra PSKVE untuk Menutup Gap dari Titik A ke
Titik B

\textbf{Penulis:} \textbf{Versi:} draf v0.1 \textbf{Catatan:} PSKVE
merujuk pada lima dimensi (P, S, K, V, E) sesuai definisi Anda; buku ini
menggunakan notasi tersebut tanpa menetapkan penamaan baku agar
kompatibel dengan berbagai domain.

\begin{center}\rule{0.5\linewidth}{0.5pt}\end{center}

\subsection{Kata Pengantar}\label{kata-pengantar}

Kita hidup di era di mana kebutuhan bernilai tinggi makin kompleks,
sementara kapasitas individu sering tak mencukupi untuk mencapainya
sendirian. \textbf{TISE (Triune‑Intelligence Smart Engineering)}
menawarkan paradigma rekayasa cerdas yang mengorkestrasi \textbf{NI
(Natural/Human Intelligence)}, \textbf{AI (Artificial Intelligence)},
dan \textbf{CI (Collective Intelligence)} di dalam \textbf{SUB‑RUANG}
pada \textbf{RUANG PSKVE} untuk menutup \textbf{gap} dari \textbf{Titik
A (keadaan kini)} ke \textbf{Titik B (keadaan target)} secara efektif,
etis, terukur, dan berkelanjutan.

Paradigma ini memetakan masalah sebagai sebuah \textbf{gap} di ruang
multi‑dimensi PSKVE, lalu membangun \textbf{lingkungan kerja} yang
terdiri dari \textbf{stasiun, jalan, dan kendaraan} yang berperan
sebagai \textbf{agen cerdas} bersiklus \textbf{PUDAL} (Perception,
Understanding, Decision, Action, Learning). Setiap agen digerakkan oleh
\textbf{CORE ENGINE} yang mengonversi \textbf{energon} (satuan
sumber‑daya) menjadi \textbf{work} (usaha nyata) melalui siklus
\textbf{input → encode → decode → output}.

Buku ini menyajikan landasan, arsitektur, pola orkestrasi, tata kelola,
hingga implementasi dan studi kasus. Sasaran pembaca meliputi praktisi,
peneliti, pendidik, pemimpin organisasi, dan pembuat kebijakan.

\begin{center}\rule{0.5\linewidth}{0.5pt}\end{center}

\subsection{Cara Menggunakan Buku Ini}\label{cara-menggunakan-buku-ini}

\begin{itemize}
\tightlist
\item
  \textbf{Jalur Praktisi:} Baca Bagian I--III, lalu Bagian V--VI untuk
  pola orkestrasi, tooling, dan studi kasus.
\item
  \textbf{Jalur Peneliti:} Dalami Bagian II \& VII mengenai CORE ENGINE,
  konversi energon, metodologi eksperimen, dan standar publikasi.
\item
  \textbf{Jalur Pendidik:} Gunakan kerangka PSKVE \& PUDAL untuk
  merancang tugas, rubrik, dan \emph{learning analytics}. Lihat Lampiran
  C.
\end{itemize}

Setiap bab dilengkapi \textbf{Ringkasan}, \textbf{Tujuan Pembelajaran},
\textbf{Pertanyaan Pemandu}, \textbf{Peta Konsep}, \textbf{Latihan}, dan
\textbf{Kartu Aksi}.

\begin{center}\rule{0.5\linewidth}{0.5pt}\end{center}

\subsection{Notasi Inti}\label{notasi-inti}

\begin{itemize}
\tightlist
\item
  \textbf{PSKVE}: Vektor keadaan ( s,t \in \mathbb{R}\^{}5 ) →
  \textbf{gap} ( g=t-s ).
\item
  \textbf{SUB‑RUANG}: Subset operasional dalam RUANG PSKVE tempat
  pertukaran \& transformasi PSKVE.
\item
  \textbf{Agen Cerdas}: Stasiun/Jalan/Kendaraan ber‑PUDAL.
\item
  \textbf{Energon}: Satuan sumber‑daya (data, waktu, perhatian, dana,
  material, kepercayaan, kompetensi, dll.).
\item
  \textbf{CORE ENGINE}: Mesin konversi energon → work.
\end{itemize}

\begin{center}\rule{0.5\linewidth}{0.5pt}\end{center}

\section{BAGIAN I --- LANDASAN TISE}\label{bagian-i-landasan-tise}

\subsection{Bab 1. Narasi Masalah \& Ruang
PSKVE}\label{bab-1.-narasi-masalah-ruang-pskve}

\textbf{Ringkasan.} Masalah direpresentasikan sebagai \textbf{gap}
antara kondisi sekarang (A) dan target (B) di RUANG PSKVE. Tujuan
rekayasa: menutup gap dengan memaksimalkan nilai multipihak dan
keberlanjutan.

\textbf{Tujuan Pembelajaran}

\begin{enumerate}
\def\labelenumi{\arabic{enumi}.}
\tightlist
\item
  Memetakan masalah ke vektor PSKVE.
\item
  Mendefinisikan target dan metrik gap.
\item
  Mengidentifikasi stakeholders dan kebutuhannya.
\end{enumerate}

\textbf{Peta Konsep}

\begin{Shaded}
\begin{Highlighting}[]
\NormalTok{graph LR}
\NormalTok{  A[Titik A] {-}{-} gap g {-}{-}\textgreater{} B[Titik B]}
\NormalTok{  subgraph RUANG PSKVE}
\NormalTok{  A}
\NormalTok{  B}
\NormalTok{  end}
\end{Highlighting}
\end{Shaded}

\textbf{Konten Utama}

\begin{itemize}
\tightlist
\item
  \textbf{Model Keadaan}: ( s=(p,s,k,v,e) ), (
  t=(p\textsuperscript{\emph{,s\^{}},k}\emph{,v\^{}},e\^{}*) ), gap (
  g=t-s ).
\item
  \textbf{Kriteria Keberhasilan}: \emph{gap‑closure rate}, kepuasan
  stakeholder, efisiensi energon, dan indeks keberlanjutan.
\item
  \textbf{Masalah sebagai Orkestrasi}: Bukan sekadar optimasi lokal,
  melainkan koordinasi lintas agen \& dimensi PSKVE.
\end{itemize}

\textbf{Kartu Aksi}

\begin{itemize}
\tightlist
\item
  Definisikan A dan B dalam 5 dimensi PSKVE.
\item
  Pilih 3 metrik paling relevan untuk diukur mingguan.
\end{itemize}

\textbf{Latihan}

\begin{enumerate}
\def\labelenumi{\arabic{enumi}.}
\tightlist
\item
  Petakan proyek Anda ke vektor PSKVE (5 angka bernorma 0--1). 2) Hitung
  besaran (\textbar g\textbar) dan diskusikan implikasi prioritas.
\end{enumerate}

\begin{center}\rule{0.5\linewidth}{0.5pt}\end{center}

\subsection{Bab 2. Triune‑Intelligence (NI × AI ×
CI)}\label{bab-2.-triuneintelligence-ni-ai-ci}

\textbf{Ringkasan.} Keunggulan TISE terletak pada kopling
\textbf{Natural/Human Intelligence (NI)}, \textbf{Artificial
Intelligence (AI)}, dan \textbf{Collective Intelligence (CI)}. Ketiganya
membentuk \emph{loop} saling menguatkan.

\textbf{Tujuan Pembelajaran}

\begin{enumerate}
\def\labelenumi{\arabic{enumi}.}
\tightlist
\item
  Merancang peran NI--AI--CI pada proses PSKVE.
\item
  Menentukan batas kewenangan dan eskalasi keputusan.
\item
  Mendesain mekanisme umpan balik untuk CI.
\end{enumerate}

\textbf{Konten Utama}

\begin{itemize}
\tightlist
\item
  \textbf{NI}: makna, etika, empati, intuisi, \emph{tacit knowledge}.
\item
  \textbf{AI}: persepsi otomatis, pemodelan, optimasi, kopilot eksekusi.
\item
  \textbf{CI}: konsensus, tata‑kelola, \emph{peer review}, pasar
  internal, komunitas.
\item
  \textbf{Arsitektur Keputusan}: keputusan penting = NI × AI × CI →
  akurasi + akuntabilitas + kecepatan.
\end{itemize}

\textbf{Peta Peran}

\begin{Shaded}
\begin{Highlighting}[]
\NormalTok{graph TD}
\NormalTok{  NI[Human Intelligence]}
\NormalTok{  AI[Artificial Intelligence]}
\NormalTok{  CI[Collective Intelligence]}
\NormalTok{  Outcome[Keputusan \& Aksi]}
\NormalTok{  NI {-}{-}\textgreater{} Outcome}
\NormalTok{  AI {-}{-}\textgreater{} Outcome}
\NormalTok{  CI {-}{-}\textgreater{} Outcome}
\end{Highlighting}
\end{Shaded}

\textbf{Kartu Aksi}

\begin{itemize}
\tightlist
\item
  Buat matriks RACI untuk keputusan inti: kolom = \{NI, AI, CI\}, baris
  = keputusan.
\end{itemize}

\textbf{Latihan}

\begin{itemize}
\tightlist
\item
  Rancang mekanisme \emph{disagreement handling} antara NI dan AI; kapan
  CI menjadi arbitrase?
\end{itemize}

\begin{center}\rule{0.5\linewidth}{0.5pt}\end{center}

\subsection{Bab 3. Lingkungan Kerja: Stasiun, Jalan,
Kendaraan}\label{bab-3.-lingkungan-kerja-stasiun-jalan-kendaraan}

\textbf{Ringkasan.} SUB‑RUANG diisi \textbf{stasiun}
(kapabilitas/layanan), \textbf{jalan} (protokol/alur kerja), dan
\textbf{kendaraan} (agen pembawa/transformator PSKVE: manusia, tim, agen
AI, robot). Engineer bertindak sebagai \textbf{dirigen} yang
mengorkestrasi semuanya.

\textbf{Tujuan Pembelajaran}

\begin{enumerate}
\def\labelenumi{\arabic{enumi}.}
\tightlist
\item
  Mendesain SUB‑RUANG minimal‑layak (MVP) untuk proyek.
\item
  Menentukan kontrak pertukaran PSKVE.
\item
  Mengatur observabilitas arus PSKVE.
\end{enumerate}

\textbf{Konten Utama}

\begin{itemize}
\tightlist
\item
  \textbf{Stasiun}: Pengetahuan, Bukti, Nilai/Insentif,
  Keterampilan/Produksi, dll.
\item
  \textbf{Jalan}: protokol, \emph{workflow}, \emph{service level
  agreement}, antarmuka.
\item
  \textbf{Kendaraan}: peran manusia, agen AI, tim lintas fungsi.
\item
  \textbf{Kontrak PSKVE}: apa, siapa, kapan, kualitas, cara verifikasi.
\end{itemize}

\textbf{Peta Lingkungan Kerja}

\begin{Shaded}
\begin{Highlighting}[]
\NormalTok{flowchart LR}
\NormalTok{  subgraph SUB{-}RUANG}
\NormalTok{    A[Stasiun Pengetahuan]}
\NormalTok{    B[Stasiun Bukti]}
\NormalTok{    C[Stasiun Nilai/Insentif]}
\NormalTok{    D[Stasiun Produksi/Keterampilan]}
\NormalTok{    R1((Jalan/Protokol 1))}
\NormalTok{    R2((Jalan/Workflow 2))}
\NormalTok{    V1\textgreater{}Vehicle: Tim Manusia]}
\NormalTok{    V2\textgreater{}Vehicle: AI Agent]}
\NormalTok{  end}
\NormalTok{  A \textless{}{-}{-} R1 {-}{-}\textgreater{} B}
\NormalTok{  B \textless{}{-}{-} R2 {-}{-}\textgreater{} C}
\NormalTok{  C \textless{}{-}{-} R1 {-}{-}\textgreater{} D}
\NormalTok{  V1 {-}{-}{-} A}
\NormalTok{  V2 {-}{-}{-} B}
\NormalTok{  V1 {-}{-}{-} C}
\NormalTok{  V2 {-}{-}{-} D}
\end{Highlighting}
\end{Shaded}

\textbf{Kartu Aksi}

\begin{itemize}
\tightlist
\item
  Definisikan 3 stasiun awal + 2 jalan + 2 kendaraan untuk MVP.
\end{itemize}

\textbf{Latihan}

\begin{itemize}
\tightlist
\item
  Rumuskan \emph{PSKVE‑ledger} sederhana (tabel transaksi) untuk melacak
  pertukaran \& hasil.
\end{itemize}

\begin{center}\rule{0.5\linewidth}{0.5pt}\end{center}

\section{BAGIAN II --- ARSITEKTUR PUDAL \& CORE
ENGINES}\label{bagian-ii-arsitektur-pudal-core-engines}

\subsection{Bab 4. PUDAL: Siklus Agen
Cerdas}\label{bab-4.-pudal-siklus-agen-cerdas}

\begin{itemize}
\tightlist
\item
  Definisi dan peran setiap tahap; keterkaitan NI--AI--CI; pemosisian
  metrik.
\item
  Pola umum \emph{sensing → modeling → deciding → executing → learning}.
\end{itemize}

\subsection{Bab 5--9. Lima CORE ENGINE}\label{bab-59.-lima-core-engine}

\begin{itemize}
\tightlist
\item
  \textbf{Perception Engine}: akuisisi sinyal, pembersihan,
  fitur/representasi.
\item
  \textbf{Understanding Engine}: pemodelan, penjelasan, diagnosis gap.
\item
  \textbf{Decision Engine}: preferensi multipihak, MCDA, \emph{routing}
  PSKVE.
\item
  \textbf{Action Engine}: eksekusi layanan, produksi artefak, automasi.
\item
  \textbf{Learning Engine}: evaluasi, \emph{counterfactuals}, pembaruan
  aturan/model.
\end{itemize}

\textbf{Siklus Konversi Energon → Work}

\begin{enumerate}
\def\labelenumi{\arabic{enumi}.}
\tightlist
\item
  Input (source energon) → 2) Encode (core energon) → 3) Decode (working
  energon) → 4) Output (hasil \& reset).
\end{enumerate}

\textbf{KPI/KRI per Engine} (contoh): latensi sensing, kualitas model,
\emph{decision regret}, throughput aksi, \emph{learning gain}.

\begin{center}\rule{0.5\linewidth}{0.5pt}\end{center}

\section{BAGIAN III --- ORKESTRASI, GOVERNANCE, \&
SUSTAINABILITY}\label{bagian-iii-orkestrasi-governance-sustainability}

\subsection{Bab 10. Pola Orkestrasi
SUB‑RUANG}\label{bab-10.-pola-orkestrasi-subruang}

\begin{itemize}
\tightlist
\item
  \emph{Pipelines}, \emph{markets}, \emph{workcells}, \emph{hubs},
  \emph{circuits}.
\item
  Desain rute, antrian, kapasitas, dan SLA.
\end{itemize}

\subsection{Bab 11. Governance \& Etika
Triune}\label{bab-11.-governance-etika-triune}

\begin{itemize}
\tightlist
\item
  Batas kewenangan NI/AI, \emph{human‑in‑the‑loop}, auditability,
  \emph{responsible AI}.
\end{itemize}

\subsection{Bab 12. Sustainability \&
Risiko}\label{bab-12.-sustainability-risiko}

\begin{itemize}
\tightlist
\item
  Efisiensi energon, jejak sumber‑daya, ketahanan, \emph{failure modes},
  mitigasi.
\end{itemize}

\begin{center}\rule{0.5\linewidth}{0.5pt}\end{center}

\section{BAGIAN IV --- INSTRUMEN \&
TEKNOLOGI}\label{bagian-iv-instrumen-teknologi}

\subsection{Bab 13. Prinsip Konversi
Energon}\label{bab-13.-prinsip-konversi-energon}

\begin{itemize}
\tightlist
\item
  Hukum konservasi, efisiensi, rugi konversi, keamanan konversi
  (data→insight, dana→kapasitas, kepercayaan→komitmen).
\end{itemize}

\subsection{Bab 14. Desain Instrumen}\label{bab-14.-desain-instrumen}

\begin{itemize}
\tightlist
\item
  Encoder/decoder PSKVE, skema insentif, kontrak \& verifikasi bukti.
\end{itemize}

\subsection{Bab 15. Implementasi
Acuan}\label{bab-15.-implementasi-acuan}

\begin{itemize}
\tightlist
\item
  \emph{PSKVE‑ledger}, katalog stasiun, katalog kendaraan, pustaka rute,
  dasbor gap.
\end{itemize}

\subsection{Bab 16. Tooling}\label{bab-16.-tooling}

\begin{itemize}
\tightlist
\item
  Contoh \emph{stack}: Python, Prolog/ontologi, Quarto/Typst,
  Git/GitHub, dashboard.
\end{itemize}

\begin{center}\rule{0.5\linewidth}{0.5pt}\end{center}

\section{BAGIAN V --- STUDI KASUS}\label{bagian-v-studi-kasus}

\subsection{Bab 17. Smart Loan Arranger
(UMKM)}\label{bab-17.-smart-loan-arranger-umkm}

\begin{itemize}
\tightlist
\item
  Stasiun: Risiko, Bukti Transaksi, Nilai/Insentif, Produksi.
\item
  Kendaraan: Analis (NI), agen skor (AI), komunitas lender--borrower
  (CI).
\item
  Metrik: NPL, \emph{gap closure} PSKVE borrower, \emph{co‑creation
  value}.
\end{itemize}

\subsection{Bab 18. GRACE: Ekosistem
Lansia}\label{bab-18.-grace-ekosistem-lansia}

\textbf{Ringkasan.} GRACE adalah ekosistem pemberdayaan lansia yang
memfasilitasi pertukaran \& transformasi \textbf{PSKVE} lintas
\textbf{stasiun--jalan--kendaraan} dengan kopling \textbf{NI--AI--CI}.
Tujuan akhirnya: \emph{well‑being}, kemandirian, makna, dan
keberlanjutan sosial‑ekonomi. TISE memandang GRACE sebagai
\textbf{SUB‑RUANG} pada RUANG PSKVE, di mana \textbf{energon} (data,
waktu, perhatian, kepercayaan, kompetensi, dana) dikonversi menjadi
\textbf{work} (layanan nyata, partisipasi, kontribusi, outcome kesehatan
\& sosial) melalui PUDAL.

\textbf{Tujuan Pembelajaran}

\begin{enumerate}
\def\labelenumi{\arabic{enumi}.}
\tightlist
\item
  Mendesain SUB‑RUANG GRACE (stasiun, jalan, kendaraan) dan kontrak
  PSKVE.
\item
  Merancang CORE ENGINES PUDAL untuk intervensi kesejahteraan lansia.
\item
  Menetapkan metrik \emph{gap closure}, fairness, dan sustainability.
\end{enumerate}

\textbf{Peta Konsep}

\begin{Shaded}
\begin{Highlighting}[]
\NormalTok{graph LR}
\NormalTok{  A[Titik A: kondisi lansia kini] {-}{-} gap g {-}{-}\textgreater{} B[Titik B: sejahtera \& mandiri]}
\NormalTok{  subgraph SUB{-}RUANG GRACE}
\NormalTok{    M[Stasiun Misi]}
\NormalTok{    R[Stasiun Relasi]}
\NormalTok{    K[Stasiun Kontribusi]}
\NormalTok{    A2[Stasiun Aktivitas \& Kesehatan]}
\NormalTok{    E[Stasiun Ekonomi \& Dukungan]}
\NormalTok{  end}
\NormalTok{  M\textless{}{-}{-}\textgreater{}R}
\NormalTok{  R\textless{}{-}{-}\textgreater{}K}
\NormalTok{  K\textless{}{-}{-}\textgreater{}A2}
\NormalTok{  A2\textless{}{-}{-}\textgreater{}E}
\NormalTok{  classDef s fill:\#eef,stroke:\#88f;}
\NormalTok{  class M,R,K,A2,E s;}
\end{Highlighting}
\end{Shaded}

\subsubsection{18.1 Latar Belakang \& Hipotesis
Nilai}\label{latar-belakang-hipotesis-nilai}

\begin{itemize}
\tightlist
\item
  \textbf{Masalah}: Kesepian, penurunan fungsi, beban caregiver, dan
  akses layanan yang terfragmentasi.
\item
  \textbf{Hipotesis}: Orkestrasi PSKVE berbasis TISE meningkatkan
  \emph{well‑being} \& kemandirian dengan biaya yang terkendali, serta
  memperluas partisipasi komunitas.
\end{itemize}

\subsubsection{18.2 Pemetaan Stakeholder \& Peran
NI--AI--CI}\label{pemetaan-stakeholder-peran-niaici}

\begin{longtable}[]{@{}
  >{\raggedright\arraybackslash}p{(\linewidth - 6\tabcolsep) * \real{0.1610}}
  >{\raggedright\arraybackslash}p{(\linewidth - 6\tabcolsep) * \real{0.2627}}
  >{\raggedright\arraybackslash}p{(\linewidth - 6\tabcolsep) * \real{0.3390}}
  >{\raggedright\arraybackslash}p{(\linewidth - 6\tabcolsep) * \real{0.2373}}@{}}
\toprule\noalign{}
\begin{minipage}[b]{\linewidth}\raggedright
Stakeholder
\end{minipage} & \begin{minipage}[b]{\linewidth}\raggedright
Peran NI (Human)
\end{minipage} & \begin{minipage}[b]{\linewidth}\raggedright
Peran AI
\end{minipage} & \begin{minipage}[b]{\linewidth}\raggedright
Peran CI
\end{minipage} \\
\midrule\noalign{}
\endhead
\bottomrule\noalign{}
\endlastfoot
Lansia & Preferensi, makna, tujuan hidup & Coach digital, pengingat,
deteksi risiko & Kelompok dukungan, komunitas \\
Keluarga/Teman & Empati, dukungan harian & Asisten caregiving,
\emph{insight} routines & Jaringan keluarga \& tetangga \\
Caregiver/Relawan & Perawatan, pendampingan & Penjadwal, prioritas,
triase & Koordinasi shift \& rotasi \\
Faskes/Profesional & Diagnosis, terapi & CDS, risk scoring & Rujukan
terintegrasi \\
Komunitas \& Gereja & Rohani, sosial & Matching kegiatan & Forum \&
tata‑kelola \\
Pemerintah/Asuransi & Kebijakan, skema & Fraud/risk, simulasi biaya &
Konsensus pembiayaan \\
\end{longtable}

\subsubsection{18.3 Desain SUB‑RUANG GRACE}\label{desain-subruang-grace}

\textbf{Stasiun} (layanan inti \& kapabilitas):

\begin{enumerate}
\def\labelenumi{\arabic{enumi}.}
\tightlist
\item
  \textbf{Misi} --- \emph{life purpose coaching}, spiritual care, narasi
  makna.
\item
  \textbf{Relasi} --- \emph{relationship hub}, grup minat, \emph{buddy
  system}, kunjungan.
\item
  \textbf{Kontribusi} --- \emph{contribution studio}, \emph{micro‑jobs},
  mentoring, volunteering.
\item
  \textbf{Aktivitas \& Kesehatan} --- aktivitas fisik/kognitif,
  monitoring vital, intervensi rumah.
\item
  \textbf{Ekonomi \& Dukungan} --- akses bantuan, asuransi, subsidi,
  logistik (obat, alat bantu).
\end{enumerate}

\textbf{Jalan} (protokol \& workflow): onboarding, asesmen PSKVE,
penentuan rencana, rujukan antar‑stasiun, \emph{follow‑up}, eskalasi
klinis, \emph{closing the loop}.

\textbf{Kendaraan} (agens): peran manusia (pendamping, fasilitator),
agen AI (coach/triase), tim lintas fungsi, perangkat IoT, aplikasi
seluler.

\textbf{Kontrak PSKVE}: muatan (data, waktu, perhatian, dukungan, dana),
standar kualitas (respons‑time, adherence), verifikasi (peer‑review,
bukti aktivitas), \emph{privacy \& consent}.

\subsubsection{18.4 PUDAL \& CORE ENGINES per
Stasiun}\label{pudal-core-engines-per-stasiun}

\begin{itemize}
\tightlist
\item
  \textbf{Perception}: gejala, perilaku sosial, jadwal minum obat,
  \emph{wearables}, check‑in emosional.
\item
  \textbf{Understanding}: profil risiko komposit
  (fisik--kognitif--sosial--spiritual), \emph{trajectory} fungsi.
\item
  \textbf{Decision}: rute antar‑stasiun (mis. dari Relasi → Aktivitas),
  prioritas intervensi, \emph{who‑does‑what‑when}.
\item
  \textbf{Action}: kunjungan, aktivitas kelompok, intervensi rumah,
  tele‑konsultasi, \emph{micro‑jobs}.
\item
  \textbf{Learning}: \emph{outcome tracking}, \emph{counterfactuals},
  penyesuaian rencana personal.
\end{itemize}

\textbf{CORE ENGINE} menerapkan siklus \textbf{Energon → Work}: 1) Input
(mis. waktu relawan) → 2) Encode (slot jadwal \& komitmen) → 3) Decode
(kegiatan terjadwal) → 4) Output (kehadiran \& efek pada
\textbar g\textbar) + \emph{reset}.

\subsubsection{18.5 Taksonomi Energon
(contoh)}\label{taksonomi-energon-contoh}

\begin{longtable}[]{@{}
  >{\raggedright\arraybackslash}p{(\linewidth - 6\tabcolsep) * \real{0.1410}}
  >{\raggedright\arraybackslash}p{(\linewidth - 6\tabcolsep) * \real{0.2821}}
  >{\raggedright\arraybackslash}p{(\linewidth - 6\tabcolsep) * \real{0.2692}}
  >{\raggedright\arraybackslash}p{(\linewidth - 6\tabcolsep) * \real{0.3077}}@{}}
\toprule\noalign{}
\begin{minipage}[b]{\linewidth}\raggedright
Kategori
\end{minipage} & \begin{minipage}[b]{\linewidth}\raggedright
Contoh Source Energon
\end{minipage} & \begin{minipage}[b]{\linewidth}\raggedright
Core Energon (encode)
\end{minipage} & \begin{minipage}[b]{\linewidth}\raggedright
Working Energon (decode)
\end{minipage} \\
\midrule\noalign{}
\endhead
\bottomrule\noalign{}
\endlastfoot
Data & vital, mobilitas, mood & fitur terstandar & skor risiko,
rekomendasi \\
Waktu & jam relawan/caregiver & slot jadwal & kunjungan, pendampingan \\
Perhatian & sesi komunikasi & \emph{commitment token} & interaksi
bermakna \\
Kepercayaan & reputasi, referensi & \emph{trust score} & akses
preferensial \\
Kompetensi & keahlian relawan/ahli & katalog kompetensi & \emph{task
matching} \\
Dana/Barang & bantuan tunai/alat & \emph{budget token} &
pengadaan/logistik \\
\end{longtable}

\subsubsection{18.6 Ledger PSKVE \& Skema
Data}\label{ledger-pskve-skema-data}

\textbf{Contoh CSV ledger}

\begin{verbatim}
timestamp,from_station,to_station,vehicle,pskve_payload,quality,verifier,outcome
2025-01-12T09:00,Relasi,Aktivitas,Relawan,"kunjungan:30m;latihan ringan",0.9,koordinator,completed
2025-01-12T10:15,Aktivitas,Ekonomi,AI-agent,"permintaan alat bantu jalan",0.95,case-worker,approved
\end{verbatim}

\textbf{Skema inti (JSON) --- event}

\begin{Shaded}
\begin{Highlighting}[]
\FunctionTok{\{}
  \DataTypeTok{"event\_id"}\FunctionTok{:} \StringTok{"evt\_001"}\FunctionTok{,}
  \DataTypeTok{"person\_id"}\FunctionTok{:} \StringTok{"grace\_123"}\FunctionTok{,}
  \DataTypeTok{"from"}\FunctionTok{:} \StringTok{"Relasi"}\FunctionTok{,}
  \DataTypeTok{"to"}\FunctionTok{:} \StringTok{"Aktivitas"}\FunctionTok{,}
  \DataTypeTok{"vehicle"}\FunctionTok{:} \StringTok{"Relawan"}\FunctionTok{,}
  \DataTypeTok{"payload"}\FunctionTok{:} \FunctionTok{\{}\DataTypeTok{"type"}\FunctionTok{:} \StringTok{"kunjungan"}\FunctionTok{,} \DataTypeTok{"duration\_min"}\FunctionTok{:} \DecValTok{30}\FunctionTok{\},}
  \DataTypeTok{"metrics"}\FunctionTok{:} \FunctionTok{\{}\DataTypeTok{"mood\_delta"}\FunctionTok{:} \FloatTok{0.2}\FunctionTok{,} \DataTypeTok{"mobility\_delta"}\FunctionTok{:} \FloatTok{0.1}\FunctionTok{\},}
  \DataTypeTok{"quality"}\FunctionTok{:} \FloatTok{0.9}\FunctionTok{,}
  \DataTypeTok{"verifier"}\FunctionTok{:} \StringTok{"koordinator"}\FunctionTok{,}
  \DataTypeTok{"outcome"}\FunctionTok{:} \StringTok{"completed"}
\FunctionTok{\}}
\end{Highlighting}
\end{Shaded}

\subsubsection{18.7 Indikator \& Dashboard}\label{indikator-dashboard}

\begin{itemize}
\tightlist
\item
  \textbf{Gap Closure Rate} (\textbar g\textbar{} PSKVE per individu \&
  kohort)
\item
  \textbf{Independence Index} (ADL/IADL), \textbf{Loneliness Score},
  \textbf{Participation Rate}
\item
  \textbf{Caregiver Relief Hours}, \textbf{Safety Incidents} (jatuh,
  salah obat)
\item
  \textbf{Cost per Outcome}, \textbf{Sustainability Index} (efisiensi
  energon)
\item
  \textbf{Fairness Metrics} (akses setara lintas wilayah/kelompok)
\end{itemize}

\subsubsection{18.8 Rute Orkestrasi (3
skenario)}\label{rute-orkestrasi-3-skenario}

\begin{enumerate}
\def\labelenumi{\arabic{enumi}.}
\tightlist
\item
  \textbf{Kesepian → Relasi}: Perception (skor kesepian tinggi) →
  Understanding (risiko depresi) → Decision (R: \emph{buddy system} +
  komunitas minat) → Action (3 pertemuan/minggu) → Learning (turun skor
  kesepian 30\%).
\item
  \textbf{Risiko Jatuh → Intervensi}: \emph{Wearable} deteksi gait
  abnormal → triase AI → rujuk Aktivitas (latihan keseimbangan) +
  Ekonomi (alat bantu) → \emph{home modification} → insiden turun 50\%.
\item
  \textbf{Kontribusi → Micro‑jobs}: Profil kompetensi → \emph{matching}
  mentoring anak, dokumentasi sejarah lokal, kerajinan → insentif
  kecil/rekognisi → peningkatan makna \& relasi.
\end{enumerate}

\textbf{Blueprint Layanan (Mermaid)}

\begin{Shaded}
\begin{Highlighting}[]
\NormalTok{sequenceDiagram}
\NormalTok{  participant L as Lansia}
\NormalTok{  participant R as Stasiun Relasi}
\NormalTok{  participant A as Stasiun Aktivitas}
\NormalTok{  participant E as Stasiun Ekonomi}
\NormalTok{  participant AI as Agen AI}
\NormalTok{  L{-}\textgreater{}\textgreater{}R: Check{-}in (mood lonely)}
\NormalTok{  R{-}\textgreater{}\textgreater{}AI: Skor \& rekomendasi}
\NormalTok{  AI{-}{-}\textgreater{}\textgreater{}R: Rekomendasi buddy + aktivitas}
\NormalTok{  R{-}\textgreater{}\textgreater{}A: Rujukan kelas senam ringan}
\NormalTok{  A{-}\textgreater{}\textgreater{}E: Permintaan alat bantu jalan}
\NormalTok{  E{-}{-}\textgreater{}\textgreater{}L: Pengadaan \& pengantaran}
\NormalTok{  A{-}{-}\textgreater{}\textgreater{}L: Pelatihan \& pemantauan}
\NormalTok{  L{-}{-}\textgreater{}\textgreater{}R: Umpan balik (mood naik)}
\end{Highlighting}
\end{Shaded}

\subsubsection{18.9 Governance, Etika, \&
Privasi}\label{governance-etika-privasi}

\begin{itemize}
\tightlist
\item
  \textbf{Consent berlapis}, minimasi data, \emph{purpose limitation},
  \emph{role‑based access}.
\item
  \textbf{Human‑in‑the‑loop} pada keputusan material; audit
  \emph{explainability} untuk AI.
\item
  \textbf{Safeguards}: \emph{red‑flag escalation}, perlindungan kelompok
  rentan, \emph{bias check}.
\end{itemize}

\subsubsection{18.10 Rencana Pilot \& Skala}\label{rencana-pilot-skala}

\textbf{Tahap P0 (0--3 bln)}: desain layanan, rekrut koordinator, 30
peserta, 3 stasiun aktif. \textbf{P1 (4--9 bln)}: tambah 150 peserta,
integrasi \emph{wearables}, ledger operasional. \textbf{P2 (10--18
bln)}: kontrak pembiayaan, \emph{cost‑effectiveness} study, standarisasi
SOP. \textbf{P3 (18+ bln)}: perluasan lintas kecamatan/kota;
\emph{train‑the‑trainer}.

\textbf{Kebutuhan Peran}: Dirigen (engineer), case‑worker, fasilitator
relasi, pelatih aktivitas, koordinator ekonomi, \emph{data steward},
\emph{AI ops}.

\subsubsection{18.11 Metodologi Evaluasi \&
Publikasi}\label{metodologi-evaluasi-publikasi}

\begin{itemize}
\tightlist
\item
  Desain kuasi-eksperimental / \emph{stepped‑wedge}; pengukuran
  baseline--follow‑up (3/6/12 bln).
\item
  Analisis \emph{difference‑in‑differences} untuk outcome utama;
  \emph{process mining} dari ledger.
\item
  Publikasi: \emph{open instruments}, \emph{reporting checklist},
  template reproducible (Quarto).
\end{itemize}

\textbf{Kartu Aksi Bab 18}

\begin{itemize}
\tightlist
\item
  Bentuk tim inti \& pilih 30 peserta awal.
\item
  Tetapkan 5 KPI prioritas \& definisikan skema ledger.
\item
  Bangun \emph{buddy system} + 1 kelas aktivitas mingguan + 1 jalur
  bantuan ekonomi.
\end{itemize}

\textbf{Latihan Bab 18}

\begin{enumerate}
\def\labelenumi{\arabic{enumi}.}
\tightlist
\item
  Buat matriks rute antar‑stasiun untuk tiga persona (mandiri, rapuh,
  dengan komorbid).
\item
  Desain \emph{consent flow} berlapis dan daftar \emph{red flags} untuk
  eskalasi.
\end{enumerate}

\subsection{Bab 19. Smart Bed / Smart Furniture Smart Bed / Smart
Furniture}\label{bab-19.-smart-bed-smart-furniture-smart-bed-smart-furniture}

\begin{itemize}
\tightlist
\item
  PUDAL fisik‑siber; persepsi → intervensi → pembelajaran personal.
\end{itemize}

\subsection{Bab 20. Transformasi Universitas \& Kelas
Cerdas}\label{bab-20.-transformasi-universitas-kelas-cerdas}

\begin{itemize}
\tightlist
\item
  Orkestrasi PSKVE untuk kurikulum, asesmen, dan ekosistem riset.
\end{itemize}

\begin{center}\rule{0.5\linewidth}{0.5pt}\end{center}

\section{BAGIAN VI --- RISET \&
EVALUASI}\label{bagian-vi-riset-evaluasi}

\subsection{Bab 21. Metodologi
Eksperimen}\label{bab-21.-metodologi-eksperimen}

\begin{itemize}
\tightlist
\item
  Desain studi, A/B, \emph{field trials}, validasi eksternal.
\end{itemize}

\subsection{Bab 22. Metrik \& Evaluasi}\label{bab-22.-metrik-evaluasi}

\begin{itemize}
\tightlist
\item
  \textbf{Gap Closure Rate}, \textbf{PSKVE Throughput},
  \textbf{Stakeholder Utility}, \textbf{Sustainability Index},
  \textbf{Reliability/Safety}.
\end{itemize}

\subsection{Bab 23. Publikasi \&
Standar}\label{bab-23.-publikasi-standar}

\begin{itemize}
\tightlist
\item
  Template, \emph{open instruments}, repositori rujukan, lisensi.
\end{itemize}

\begin{center}\rule{0.5\linewidth}{0.5pt}\end{center}

\section{Lampiran}\label{lampiran}

\subsection{Lampiran A. Glosarium
Singkat}\label{lampiran-a.-glosarium-singkat}

\begin{itemize}
\tightlist
\item
  \textbf{TISE, PSKVE, PUDAL, Energon, CORE ENGINE, SUB‑RUANG, Stasiun,
  Jalan, Kendaraan, Ledger}
\end{itemize}

\subsection{Lampiran B. Checklist
Orkestrasi}\label{lampiran-b.-checklist-orkestrasi}

\begin{enumerate}
\def\labelenumi{\arabic{enumi}.}
\tightlist
\item
  Pemetaan A dan B (PSKVE). 2) Peta stakeholders \& NI--AI--CI. 3)
  SUB‑RUANG (stasiun, jalan, kendaraan). 4) Kontrak PSKVE. 5) PUDAL \&
  CORE ENGINE. 6) Observabilitas. 7) KPI/KRI. 8) Siklus belajar.
\end{enumerate}

\subsection{Lampiran C. Template
Praktis}\label{lampiran-c.-template-praktis}

\textbf{1) Struktur Proyek Quarto (buku):}

\begin{verbatim}
paradigma-tise/
  _quarto.yml
  index.qmd
  part1/
    01-narasi-masalah.qmd
    02-triune-intelligence.qmd
    03-lingkungan-kerja.qmd
  part2/
    04-pudal.qmd
    05-perception-engine.qmd
    06-understanding-engine.qmd
    07-decision-engine.qmd
    08-action-engine.qmd
    09-learning-engine.qmd
  part3/
    10-orkestrasi.qmd
    11-governance.qmd
    12-sustainability.qmd
  part4/
    13-energon-principles.qmd
    14-instrument-design.qmd
    15-reference-impl.qmd
    16-tooling.qmd
  part5/
    17-case-sla.qmd
    18-case-grace.qmd
    19-case-smartbed.qmd
    20-case-university.qmd
  part6/
    21-methods.qmd
    22-metrics.qmd
    23-publication.qmd
  assets/
    figures/
    tables/
    styles.css
  data/
  refs.bib
\end{verbatim}

\textbf{2) Contoh \texttt{\_quarto.yml}:}

\begin{Shaded}
\begin{Highlighting}[]
\FunctionTok{title}\KeywordTok{:}\AttributeTok{ }\StringTok{"Paradigma TISE: Orkestra PSKVE dari A ke B"}
\FunctionTok{author}\KeywordTok{:}\AttributeTok{ }\KeywordTok{[}\StringTok{"\textless{}isi nama penulis\textgreater{}"}\KeywordTok{]}
\FunctionTok{lang}\KeywordTok{:}\AttributeTok{ id}
\FunctionTok{format}\KeywordTok{:}
\AttributeTok{  }\FunctionTok{html}\KeywordTok{:}
\AttributeTok{    }\FunctionTok{theme}\KeywordTok{:}\AttributeTok{ cosmo}
\AttributeTok{    }\FunctionTok{toc}\KeywordTok{:}\AttributeTok{ }\CharTok{true}
\AttributeTok{  }\FunctionTok{pdf}\KeywordTok{:}
\AttributeTok{    }\FunctionTok{documentclass}\KeywordTok{:}\AttributeTok{ scrreprt}
\FunctionTok{editor}\KeywordTok{:}\AttributeTok{ visual}
\FunctionTok{number{-}sections}\KeywordTok{:}\AttributeTok{ }\CharTok{true}
\FunctionTok{bibliography}\KeywordTok{:}\AttributeTok{ refs.bib}
\FunctionTok{site{-}url}\KeywordTok{:}\AttributeTok{ }\StringTok{""}
\end{Highlighting}
\end{Shaded}

\textbf{3) Contoh awal \texttt{01-narasi-masalah.qmd}:}

\begin{Shaded}
\begin{Highlighting}[]
\CommentTok{{-}{-}{-}}
\AnnotationTok{title:}\CommentTok{ "Narasi Masalah \& Ruang PSKVE"}
\CommentTok{{-}{-}{-}}

\NormalTok{::: \{.callout{-}note\}}
\NormalTok{**Tujuan:** Memetakan masalah ke vektor PSKVE dan merancang metrik gap.}
\NormalTok{:::}

\NormalTok{Masalah bernilai tinggi kita modelkan sebagai gap antara keadaan kini (A) dan target (B) di RUANG PSKVE. Dengan menyatakan keduanya sebagai vektor berdimensi lima, kita dapat menilai, memprioritaskan, dan mengorkestrasi jalan menuju B secara terukur.}

\InformationTok{\textasciigrave{}\textasciigrave{}\textasciigrave{}\{mermaid\}}
\NormalTok{graph LR}
\NormalTok{  A[Titik A] {-}{-} g {-}{-}\textgreater{} B[Titik B]}
\end{Highlighting}
\end{Shaded}

\textbf{Kartu Aksi}: Definisikan A, B, serta tiga metrik yang dipantau
mingguan.

\begin{verbatim}

**4) Template Ledger PSKVE (CSV):**
\end{verbatim}

timestamp,from\_station,to\_station,vehicle,pskve\_payload,quality,verifier,outcome
2025-01-01T09:00,pengetahuan,bukti,AI-agent,``fitur:
\{x1:\ldots,x2:\ldots\}'',0.92,peer-review,accepted

\begin{verbatim}

**5) Rubrik Evaluasi Gap Closure (CSV):**
\end{verbatim}

indikator,definisi,skala,bobot GCR,``laju penurunan
\textbar\textbar g\textbar\textbar{} per periode'',0-1,0.35
PT,``throughput transaksi PSKVE'',0-1,0.25 SU,``kepuasan
stakeholder'',0-1,0.25 SI,``indeks keberlanjutan'',0-1,0.15

\begin{verbatim}

## Lampiran D. Formalisasi Ringkas
- **State & Target:** \( s,t \in \mathbb{R}^5 \), **gap** \( g=t-s \).
- **Rute Orkestrasi:** \( \mathcal{R} = \{(station_i, road_{i\to i+1}, vehicle_i)\}_{i=1..n} \).
- **Konversi Energon:** \( E^{core}_i=\mathrm{encode}(E^{in}_i),\; E^{work}_i=\mathrm{decode}(E^{core}_i) \).
- **Optimasi:** Maksimalkan *gap‑closure rate* dengan batasan biaya, risiko, fairness, keberlanjutan.

## Lampiran E. Peta Orkestra TISE
```mermaid
graph LR
  Dirigen((Engineer))
  Score[Protokol/Standar]
  Stage[SUB-RUANG]
  Players[Stasiun & Kendaraan]
  Audience[Publik/Stakeholders]
  Dirigen --> Players
  Score --> Players
  Stage --> Players
  Players --> Audience
  Audience -- Umpan balik --> Dirigen
\end{verbatim}

\begin{center}\rule{0.5\linewidth}{0.5pt}\end{center}

\section{Lampiran F. Artefak Implementasi
GRACE}\label{lampiran-f.-artefak-implementasi-grace}

Lampiran ini merinci artefak siap‑pakai untuk pilot dan skala GRACE.
Setiap artefak berformat markdown/YAML/JSON/CSV agar mudah
diintegrasikan ke repositori (Git) dan alat kolaborasi.

\subsection{F.1 Daftar Artefak}\label{f.1-daftar-artefak}

\begin{enumerate}
\def\labelenumi{\arabic{enumi}.}
\tightlist
\item
  \textbf{SOP Operasional (F.2)}
\item
  \textbf{Template Formulir \& Dokumen (F.3)}
\item
  \textbf{Model Data \& Skema Database (F.4)}
\item
  \textbf{API Reference \& JSON Schemas (F.5)}
\item
  \textbf{Data Pipeline \& Integrasi Perangkat (F.6)}
\item
  \textbf{Dashboard \& Laporan (F.7)}
\item
  \textbf{QA, Uji Coba \& Monitoring (F.8)}
\item
  \textbf{Kurikulum Pelatihan (F.9)}
\item
  \textbf{Anggaran \& Resource Plan (F.10)}
\item
  \textbf{Keamanan \& Kepatuhan (F.11)}
\item
  \textbf{Checklist Eksekusi (F.12)}
\end{enumerate}

\begin{center}\rule{0.5\linewidth}{0.5pt}\end{center}

\subsection{F.2 SOP Operasional
(ringkas)}\label{f.2-sop-operasional-ringkas}

\begin{quote}
Struktur setiap SOP: \textbf{Tujuan • Lingkup • Peran (RACI) • Prasyarat
• Langkah • KPI • Formulir Terkait • Catatan}
\end{quote}

\subsubsection{SOP‑01 Onboarding
Peserta}\label{sop01-onboarding-peserta}

\begin{itemize}
\tightlist
\item
  \textbf{Tujuan:} Mendaftarkan lansia \& keluarga dengan \emph{consent}
  sah.
\item
  \textbf{RACI:} R=Case Worker, A=Koordinator GRACE, C=Data Steward,
  I=Dokter.
\item
  \textbf{Langkah:} (1) Pra‑screening → (2) Edukasi program → (3) Tanda
  tangan consent → (4) Asesmen awal (PSKVE baseline) → (5) Buat rencana
  personal.
\item
  \textbf{KPI:} \emph{Time‑to‑activate}, kelengkapan data, tingkat
  pembatalan.
\end{itemize}

\subsubsection{SOP‑02 Onboarding
Relawan}\label{sop02-onboarding-relawan}

\begin{itemize}
\tightlist
\item
  \textbf{Tujuan:} Merekrut \& mensahkan relawan.
\item
  \textbf{Langkah:} Edukasi → Pemeriksaan latar → Pelatihan →
  Penandatanganan kode etik → Penjadwalan.
\end{itemize}

\subsubsection{SOP‑03 Asesmen Awal \& Rencana
Personal}\label{sop03-asesmen-awal-rencana-personal}

\begin{itemize}
\tightlist
\item
  \textbf{Langkah:} Kuesioner ADL/IADL, \emph{loneliness}, mobilitas,
  spiritual care; tentukan target PSKVE \& intervensi.
\end{itemize}

\subsubsection{SOP‑04 Rujukan Antar
Stasiun}\label{sop04-rujukan-antar-stasiun}

\begin{itemize}
\tightlist
\item
  \textbf{Langkah:} Trigger → Validasi → Penetapan rute → Penugasan
  kendaraan (relawan/AI/tim) → Jadwal.
\end{itemize}

\subsubsection{SOP‑05 Kunjungan Rumah / Kelas
Aktivitas}\label{sop05-kunjungan-rumah-kelas-aktivitas}

\begin{itemize}
\tightlist
\item
  \textbf{Langkah:} Persiapan → Pelaksanaan → Pencatatan ledger → Umpan
  balik.
\end{itemize}

\subsubsection{SOP‑06 Eskalasi Klinis / Red
Flags}\label{sop06-eskalasi-klinis-red-flags}

\begin{itemize}
\tightlist
\item
  \textbf{Contoh Red Flags:} jatuh, kebingungan akut, penolakan
  makan/obat, ide bunuh diri.
\item
  \textbf{Langkah:} Deteksi → Konfirmasi → Hubungi PIC medis →
  Dokumentasi \& \emph{post‑incident review}.
\end{itemize}

\subsubsection{SOP‑07 Bantuan Ekonomi \&
Logistik}\label{sop07-bantuan-ekonomi-logistik}

\begin{itemize}
\tightlist
\item
  \textbf{Langkah:} Verifikasi kebutuhan → Otorisasi → Pengadaan →
  Pengantaran → Verifikasi penerimaan.
\end{itemize}

\subsubsection{SOP‑08 Proteksi Data \&
Privasi}\label{sop08-proteksi-data-privasi}

\begin{itemize}
\tightlist
\item
  \textbf{Langkah:} Minimasi data, \emph{purpose limitation}, enkripsi
  \emph{at rest} \& \emph{in transit}, kontrol akses berbasis peran,
  \emph{privacy impact assessment}.
\end{itemize}

\subsubsection{SOP‑09 Incident Response}\label{sop09-incident-response}

\begin{itemize}
\tightlist
\item
  \textbf{Langkah:} Deteksi → Triage (P1‑P3) → Mitigasi → Komunikasi
  pemangku kepentingan → Forensik → \emph{Lessons learned}.
\end{itemize}

\subsubsection{SOP‑10 Offboarding}\label{sop10-offboarding}

\begin{itemize}
\tightlist
\item
  \textbf{Langkah:} Konfirmasi penutupan → Ekspor ringkasan data →
  \emph{Right to be forgotten} → Survei kepuasan.
\end{itemize}

\begin{center}\rule{0.5\linewidth}{0.5pt}\end{center}

\subsection{F.3 Template Formulir \&
Dokumen}\label{f.3-template-formulir-dokumen}

\subsubsection{F.3.1 Informed Consent
(ID)}\label{f.3.1-informed-consent-id}

\begin{verbatim}
Judul: Persetujuan Bermaklumat (Informed Consent) Program GRACE
Pihak: [Nama Peserta] / [Wali], [Organisasi Penyelenggara]
Ruang Lingkup Data: identitas, kontak, kesehatan ringkas, aktivitas, catatan kunjungan
Tujuan Penggunaan: layanan GRACE, evaluasi program, peningkatan mutu
Hak Peserta: akses data, perbaikan, penarikan persetujuan, penghapusan data
Keamanan: enkripsi, kontrol peran, audit
Kontak Pengaduan: [email/telepon]
Tanda Tangan: ________ Tanggal: ____
\end{verbatim}

\subsubsection{F.3.2 Intake/Asesmen Awal
(YAML)}\label{f.3.2-intakeasesmen-awal-yaml}

\begin{Shaded}
\begin{Highlighting}[]
\FunctionTok{participant\_id}\KeywordTok{:}\AttributeTok{ grace\_123}
\FunctionTok{profile}\KeywordTok{:}
\AttributeTok{  }\FunctionTok{name}\KeywordTok{:}\AttributeTok{ }\StringTok{"\textless{}nama\textgreater{}"}
\AttributeTok{  }\FunctionTok{age}\KeywordTok{:}\AttributeTok{ }\DecValTok{72}
\AttributeTok{  }\FunctionTok{living\_situation}\KeywordTok{:}\AttributeTok{ }\StringTok{"sendiri|dengan keluarga|panti"}
\FunctionTok{pskve\_baseline}\KeywordTok{:}
\AttributeTok{  }\FunctionTok{purpose}\KeywordTok{:}\AttributeTok{ }\FloatTok{0.6}
\AttributeTok{  }\FunctionTok{social}\KeywordTok{:}\AttributeTok{ }\FloatTok{0.3}
\AttributeTok{  }\FunctionTok{knowledge}\KeywordTok{:}\AttributeTok{ }\FloatTok{0.4}
\AttributeTok{  }\FunctionTok{value}\KeywordTok{:}\AttributeTok{ }\FloatTok{0.5}
\AttributeTok{  }\FunctionTok{economy}\KeywordTok{:}\AttributeTok{ }\FloatTok{0.4}
\FunctionTok{risk}\KeywordTok{:}
\AttributeTok{  }\FunctionTok{fall}\KeywordTok{:}\AttributeTok{ medium}
\AttributeTok{  }\FunctionTok{loneliness}\KeywordTok{:}\AttributeTok{ high}
\FunctionTok{plan\_goals}\KeywordTok{:}
\AttributeTok{  }\KeywordTok{{-}}\AttributeTok{ }\FunctionTok{goal}\KeywordTok{:}\AttributeTok{ }\StringTok{"turunkan kesepian 30\% dalam 3 bulan"}
\AttributeTok{    }\FunctionTok{kpi}\KeywordTok{:}\AttributeTok{ }\StringTok{"UCLA{-}LS delta \textgreater{}= 0.3"}
\end{Highlighting}
\end{Shaded}

\subsubsection{F.3.3 Log Check‑in Harian
(CSV)}\label{f.3.3-log-checkin-harian-csv}

\begin{verbatim}
date,participant_id,mood,energy,med_adherence,notes
2025-01-12,grace_123,3,4,yes,"tidur cukup"
\end{verbatim}

\subsubsection{F.3.4 Laporan Kunjungan
(Markdown)}\label{f.3.4-laporan-kunjungan-markdown}

\begin{verbatim}
# Visit Report
Participant: grace_123 | Date: 2025-01-15 | Visitor: rel_045
Agenda: latihan ringan & percakapan bermakna
Observasi: stabil; mood meningkat
Tindak lanjut: kelas senam Jumat, cek alat bantu jalan
\end{verbatim}

\subsubsection{F.3.5 Form Red Flags
(Checklist)}\label{f.3.5-form-red-flags-checklist}

\begin{itemize}
\tightlist
\item[$\square$]
  Jatuh / hampir jatuh
\item[$\square$]
  Bingung berat
\item[$\square$]
  Menolak makan/obat
\item[$\square$]
  Ide bunuh diri
\item[$\square$]
  Kekerasan/KDRT
\end{itemize}

\begin{center}\rule{0.5\linewidth}{0.5pt}\end{center}

\subsection{F.4 Model Data \& Skema
Database}\label{f.4-model-data-skema-database}

\subsubsection{F.4.1 Entitas Inti}\label{f.4.1-entitas-inti}

\begin{itemize}
\tightlist
\item
  \textbf{person}(participant, caregiver), \textbf{household},
  \textbf{station}, \textbf{plan}, \textbf{plan\_item}, \textbf{event},
  \textbf{ledger}, \textbf{assessment}, \textbf{metric\_snapshot},
  \textbf{device}, \textbf{device\_reading}, \textbf{consent},
  \textbf{user}, \textbf{role}, \textbf{assignment},
  \textbf{escalation\_ticket}, \textbf{audit\_log}.
\end{itemize}

\subsubsection{F.4.2 Skema SQL (PostgreSQL ---
cuplikan)}\label{f.4.2-skema-sql-postgresql-cuplikan}

\begin{Shaded}
\begin{Highlighting}[]
\KeywordTok{CREATE} \KeywordTok{TABLE}\NormalTok{ person (}
  \KeywordTok{id}\NormalTok{ TEXT }\KeywordTok{PRIMARY} \KeywordTok{KEY}\NormalTok{,}
\NormalTok{  full\_name TEXT }\KeywordTok{NOT} \KeywordTok{NULL}\NormalTok{,}
\NormalTok{  birth\_date }\DataTypeTok{DATE}\NormalTok{,}
  \KeywordTok{role}\NormalTok{ TEXT }\KeywordTok{CHECK}\NormalTok{ (}\KeywordTok{role} \KeywordTok{IN}\NormalTok{ (}\StringTok{\textquotesingle{}participant\textquotesingle{}}\NormalTok{,}\StringTok{\textquotesingle{}caregiver\textquotesingle{}}\NormalTok{)),}
\NormalTok{  contact JSONB,}
\NormalTok{  created\_at }\DataTypeTok{TIMESTAMP} \KeywordTok{DEFAULT}\NormalTok{ now()}
\NormalTok{);}

\KeywordTok{CREATE} \KeywordTok{TABLE}\NormalTok{ station (}
  \KeywordTok{id}\NormalTok{ TEXT }\KeywordTok{PRIMARY} \KeywordTok{KEY}\NormalTok{,}
\NormalTok{  name TEXT }\KeywordTok{NOT} \KeywordTok{NULL}\NormalTok{,}
  \KeywordTok{category}\NormalTok{ TEXT }\KeywordTok{CHECK}\NormalTok{ (}\KeywordTok{category} \KeywordTok{IN}\NormalTok{ (}\StringTok{\textquotesingle{}Misi\textquotesingle{}}\NormalTok{,}\StringTok{\textquotesingle{}Relasi\textquotesingle{}}\NormalTok{,}\StringTok{\textquotesingle{}Kontribusi\textquotesingle{}}\NormalTok{,}\StringTok{\textquotesingle{}Aktivitas\textquotesingle{}}\NormalTok{,}\StringTok{\textquotesingle{}Ekonomi\textquotesingle{}}\NormalTok{))}
\NormalTok{);}

\KeywordTok{CREATE} \KeywordTok{TABLE}\NormalTok{ ledger\_event (}
  \KeywordTok{id}\NormalTok{ TEXT }\KeywordTok{PRIMARY} \KeywordTok{KEY}\NormalTok{,}
\NormalTok{  person\_id TEXT }\KeywordTok{REFERENCES}\NormalTok{ person(}\KeywordTok{id}\NormalTok{),}
\NormalTok{  from\_station TEXT }\KeywordTok{REFERENCES}\NormalTok{ station(}\KeywordTok{id}\NormalTok{),}
\NormalTok{  to\_station TEXT }\KeywordTok{REFERENCES}\NormalTok{ station(}\KeywordTok{id}\NormalTok{),}
\NormalTok{  vehicle TEXT,}
\NormalTok{  payload JSONB,}
\NormalTok{  quality }\DataTypeTok{NUMERIC} \KeywordTok{CHECK}\NormalTok{ (quality }\KeywordTok{BETWEEN} \DecValTok{0} \KeywordTok{AND} \DecValTok{1}\NormalTok{),}
\NormalTok{  verifier TEXT,}
\NormalTok{  outcome TEXT,}
\NormalTok{  ts }\DataTypeTok{TIMESTAMP} \KeywordTok{DEFAULT}\NormalTok{ now()}
\NormalTok{);}

\KeywordTok{CREATE} \KeywordTok{TABLE}\NormalTok{ consent (}
  \KeywordTok{id}\NormalTok{ TEXT }\KeywordTok{PRIMARY} \KeywordTok{KEY}\NormalTok{,}
\NormalTok{  person\_id TEXT }\KeywordTok{REFERENCES}\NormalTok{ person(}\KeywordTok{id}\NormalTok{),}
  \KeywordTok{scope}\NormalTok{ TEXT,}
\NormalTok{  granted\_at }\DataTypeTok{TIMESTAMP}\NormalTok{,}
\NormalTok{  revoked\_at }\DataTypeTok{TIMESTAMP}
\NormalTok{);}

\KeywordTok{CREATE} \KeywordTok{TABLE}\NormalTok{ metric\_snapshot (}
  \KeywordTok{id}\NormalTok{ TEXT }\KeywordTok{PRIMARY} \KeywordTok{KEY}\NormalTok{,}
\NormalTok{  person\_id TEXT }\KeywordTok{REFERENCES}\NormalTok{ person(}\KeywordTok{id}\NormalTok{),}
\NormalTok{  metric TEXT,}
  \FunctionTok{value} \DataTypeTok{NUMERIC}\NormalTok{,}
\NormalTok{  ts }\DataTypeTok{TIMESTAMP} \KeywordTok{DEFAULT}\NormalTok{ now()}
\NormalTok{);}
\end{Highlighting}
\end{Shaded}

\subsubsection{F.4.3 Data Dictionary
(cuplikan)}\label{f.4.3-data-dictionary-cuplikan}

\begin{itemize}
\tightlist
\item
  \textbf{ledger\_event.payload}: objek JSON berisi atribut domain
  (\emph{visit}, \emph{class}, \emph{device\_alert}, dll.).
\item
  \textbf{metric\_snapshot.metric}:
  \{``UCLA\_LS'',``ADL'',``IADL'',``Mobility'',``IndependenceIndex''\}.
\end{itemize}

\begin{center}\rule{0.5\linewidth}{0.5pt}\end{center}

\subsection{F.5 API Reference \& JSON
Schemas}\label{f.5-api-reference-json-schemas}

\subsubsection{F.5.1 Endpoints (REST,
v1)}\label{f.5.1-endpoints-rest-v1}

\begin{itemize}
\tightlist
\item
  \textbf{POST /v1/ledger/events} --- catat peristiwa; \emph{idempotent
  key} opsional.
\item
  \textbf{GET /v1/persons/\{id\}/metrics} --- \emph{timeseries} metrik.
\item
  \textbf{POST /v1/checkins} --- check‑in harian.
\item
  \textbf{POST /v1/consents} --- buat/ubah status consent.
\item
  \textbf{GET /v1/routes/recommend} --- rekomendasi rute antar stasiun.
\item
  \textbf{GET /v1/stations} --- daftar stasiun aktif.
\end{itemize}

\subsubsection{\texorpdfstring{F.5.2 JSON Schema ---
\texttt{LedgerEvent}}{F.5.2 JSON Schema --- LedgerEvent}}\label{f.5.2-json-schema-ledgerevent}

\begin{Shaded}
\begin{Highlighting}[]
\FunctionTok{\{}
  \DataTypeTok{"$schema"}\FunctionTok{:} \StringTok{"https://json{-}schema.org/draft/2020{-}12/schema"}\FunctionTok{,}
  \DataTypeTok{"title"}\FunctionTok{:} \StringTok{"LedgerEvent"}\FunctionTok{,}
  \DataTypeTok{"type"}\FunctionTok{:} \StringTok{"object"}\FunctionTok{,}
  \DataTypeTok{"required"}\FunctionTok{:} \OtherTok{[}\StringTok{"person\_id"}\OtherTok{,}\StringTok{"to"}\OtherTok{,}\StringTok{"vehicle"}\OtherTok{,}\StringTok{"ts"}\OtherTok{]}\FunctionTok{,}
  \DataTypeTok{"properties"}\FunctionTok{:} \FunctionTok{\{}
    \DataTypeTok{"id"}\FunctionTok{:} \FunctionTok{\{}\DataTypeTok{"type"}\FunctionTok{:} \StringTok{"string"}\FunctionTok{\},}
    \DataTypeTok{"person\_id"}\FunctionTok{:} \FunctionTok{\{}\DataTypeTok{"type"}\FunctionTok{:} \StringTok{"string"}\FunctionTok{\},}
    \DataTypeTok{"from"}\FunctionTok{:} \FunctionTok{\{}\DataTypeTok{"type"}\FunctionTok{:} \StringTok{"string"}\FunctionTok{\},}
    \DataTypeTok{"to"}\FunctionTok{:} \FunctionTok{\{}\DataTypeTok{"type"}\FunctionTok{:} \StringTok{"string"}\FunctionTok{\},}
    \DataTypeTok{"vehicle"}\FunctionTok{:} \FunctionTok{\{}\DataTypeTok{"type"}\FunctionTok{:} \StringTok{"string"}\FunctionTok{\},}
    \DataTypeTok{"payload"}\FunctionTok{:} \FunctionTok{\{}\DataTypeTok{"type"}\FunctionTok{:} \StringTok{"object"}\FunctionTok{\},}
    \DataTypeTok{"quality"}\FunctionTok{:} \FunctionTok{\{}\DataTypeTok{"type"}\FunctionTok{:} \StringTok{"number"}\FunctionTok{,} \DataTypeTok{"minimum"}\FunctionTok{:} \DecValTok{0}\FunctionTok{,} \DataTypeTok{"maximum"}\FunctionTok{:} \DecValTok{1}\FunctionTok{\},}
    \DataTypeTok{"verifier"}\FunctionTok{:} \FunctionTok{\{}\DataTypeTok{"type"}\FunctionTok{:} \StringTok{"string"}\FunctionTok{\},}
    \DataTypeTok{"outcome"}\FunctionTok{:} \FunctionTok{\{}\DataTypeTok{"type"}\FunctionTok{:} \StringTok{"string"}\FunctionTok{\},}
    \DataTypeTok{"ts"}\FunctionTok{:} \FunctionTok{\{}\DataTypeTok{"type"}\FunctionTok{:} \StringTok{"string"}\FunctionTok{,} \DataTypeTok{"format"}\FunctionTok{:} \StringTok{"date{-}time"}\FunctionTok{\}}
  \FunctionTok{\}}
\FunctionTok{\}}
\end{Highlighting}
\end{Shaded}

\begin{center}\rule{0.5\linewidth}{0.5pt}\end{center}

\subsection{F.6 Data Pipeline \& Integrasi
Perangkat}\label{f.6-data-pipeline-integrasi-perangkat}

\subsubsection{F.6.1 Arsitektur Aliran Data
(Mermaid)}\label{f.6.1-arsitektur-aliran-data-mermaid}

\begin{Shaded}
\begin{Highlighting}[]
\NormalTok{flowchart LR}
\NormalTok{  subgraph Edge}
\NormalTok{    Wearables{-}{-}\textgreater{}Phone[App Mobile]}
\NormalTok{  end}
\NormalTok{  Phone{-}{-}\textgreater{}Ingest[API Ingest]}
\NormalTok{  Ingest{-}{-}\textgreater{}Queue[Message Queue]}
\NormalTok{  Queue{-}{-}\textgreater{}Proc[Stream Processor]}
\NormalTok{  Proc{-}{-}\textgreater{}DB[(Operational DB)]}
\NormalTok{  Proc{-}{-}\textgreater{}Wh[(Data Warehouse)]}
\NormalTok{  DB{-}{-}\textgreater{}Dash[Dashboard]}
\NormalTok{  Wh{-}{-}\textgreater{}Dash}
\end{Highlighting}
\end{Shaded}

\subsubsection{F.6.2 Kebijakan Data}\label{f.6.2-kebijakan-data}

\begin{itemize}
\tightlist
\item
  \textbf{PII Segregation}, enkripsi, retensi 24 bulan, \emph{access
  logging}, \emph{least privilege}.
\end{itemize}

\begin{center}\rule{0.5\linewidth}{0.5pt}\end{center}

\subsection{F.7 Dashboard \& Laporan}\label{f.7-dashboard-laporan}

\subsubsection{F.7.1 Widget Inti}\label{f.7.1-widget-inti}

\begin{enumerate}
\def\labelenumi{\arabic{enumi}.}
\tightlist
\item
  \textbf{Gap Closure} per individu/kohort (\textbar g\textbar{} PSKVE).
\item
  \textbf{Independence Index} (ADL/IADL).
\item
  \textbf{Loneliness Score} (UCLA‑LS) \& partisipasi.
\item
  \textbf{Insiden Keamanan} \& \emph{time‑to‑respond}.
\item
  \textbf{Throughput Ledger} per stasiun \& kendaraan.
\item
  \textbf{Fairness}: akses \& outcome lintas wilayah/kelompok.
\item
  \textbf{Cost per Outcome}.
\end{enumerate}

\subsubsection{F.7.2 Kueri (pseudo‑SQL)}\label{f.7.2-kueri-pseudosql}

\begin{Shaded}
\begin{Highlighting}[]
\CommentTok{{-}{-} Gap closure terakhir 90 hari\textbackslash{} nSELECT person\_id,}
  \FunctionTok{first\_value}\NormalTok{(}\FunctionTok{value}\NormalTok{) }\KeywordTok{OVER}\NormalTok{ w }\KeywordTok{AS} \KeywordTok{start}\NormalTok{,}
  \FunctionTok{last\_value}\NormalTok{(}\FunctionTok{value}\NormalTok{) }\KeywordTok{OVER}\NormalTok{ w }\KeywordTok{AS} \ControlFlowTok{end}\NormalTok{,}
\NormalTok{  (}\ControlFlowTok{end}\OperatorTok{{-}}\KeywordTok{start}\NormalTok{) }\KeywordTok{AS}\NormalTok{ delta}
\KeywordTok{FROM}\NormalTok{ metric\_snapshot}
\KeywordTok{WHERE}\NormalTok{ metric}\OperatorTok{=}\StringTok{\textquotesingle{}IndependenceIndex\textquotesingle{}}
\NormalTok{WINDOW w }\KeywordTok{AS}\NormalTok{ (}\KeywordTok{PARTITION} \KeywordTok{BY}\NormalTok{ person\_id }\KeywordTok{ORDER} \KeywordTok{BY}\NormalTok{ ts}
  \KeywordTok{RANGE} \KeywordTok{BETWEEN} \DataTypeTok{INTERVAL} \StringTok{\textquotesingle{}90 days\textquotesingle{}} \KeywordTok{PRECEDING} \KeywordTok{AND} \KeywordTok{CURRENT} \KeywordTok{ROW}\NormalTok{);}
\end{Highlighting}
\end{Shaded}

\begin{center}\rule{0.5\linewidth}{0.5pt}\end{center}

\subsection{F.8 QA, Uji Coba \&
Monitoring}\label{f.8-qa-uji-coba-monitoring}

\begin{itemize}
\tightlist
\item
  \textbf{Test Plan:} unit (API), integrasi (edge→warehouse), UAT (SOP
  alur), beban (1k peserta).
\item
  \textbf{Acceptance:} kelengkapan event ≥ 98\%, \emph{SLO ingest
  latency} p95 \textless{} 2s, \emph{error rate} \textless{} 0.5\%.
\item
  \textbf{Monitoring:} \emph{health checks}, \emph{synthetic
  transactions}, \emph{alerting} on-call.
\end{itemize}

\begin{center}\rule{0.5\linewidth}{0.5pt}\end{center}

\subsection{F.9 Kurikulum Pelatihan}\label{f.9-kurikulum-pelatihan}

\begin{itemize}
\tightlist
\item
  \textbf{Relawan (6 jam):} etika \& privasi, komunikasi empatik, red
  flags, pencatatan ledger.
\item
  \textbf{Case Worker (8 jam):} asesmen PSKVE, rute antar stasiun,
  eskalasi.
\item
  \textbf{Koordinator (6 jam):} penjadwalan, QA, pelaporan.
\item
  \textbf{Data Steward \& AI Ops (8 jam):} kebijakan data, data quality,
  model monitoring.
\item
  \textbf{Evaluasi:} kuis + \emph{observed structured practice}.
\end{itemize}

\begin{center}\rule{0.5\linewidth}{0.5pt}\end{center}

\subsection{F.10 Anggaran \& Resource
Plan}\label{f.10-anggaran-resource-plan}

\subsubsection{F.10.1 Template Anggaran
(CSV)}\label{f.10.1-template-anggaran-csv}

\begin{verbatim}
item,kategori,kuantitas,satuan,harga_satuan,total,catatan
Koordinator,OPEX,1,orang/bulan,12000000,12000000,
Relawan insentif,OPEX,30,orang/bulan,300000,9000000,
Wearables,CAPEX,20,unit,800000,16000000,
Server & Cloud,OPEX,1,bulan,6000000,6000000,
Pelatihan,OPEX,2,sesi,2500000,5000000,
\end{verbatim}

\subsubsection{F.10.2 Rencana Peran (RACI
ringkas)}\label{f.10.2-rencana-peran-raci-ringkas}

\begin{itemize}
\tightlist
\item
  \textbf{Dirigen/Engineer}, \textbf{Koordinator}, \textbf{Case Worker},
  \textbf{Relawan}, \textbf{Data Steward}, \textbf{AI Ops},
  \textbf{Dokter/Perawat}.
\end{itemize}

\begin{center}\rule{0.5\linewidth}{0.5pt}\end{center}

\subsection{F.11 Keamanan \& Kepatuhan}\label{f.11-keamanan-kepatuhan}

\begin{itemize}
\tightlist
\item
  \textbf{Matriks Akses Berbasis Peran (RBAC)}: \emph{participant}
  (self‑access), \emph{caregiver} (delegated), \emph{worker},
  \emph{coordinator}, \emph{steward}, \emph{admin}.
\item
  \textbf{Klasifikasi Data:} Publik, Internal, Rahasia, Sensitif
  Kesehatan.
\item
  \textbf{Audit \& Log:} \emph{immutable logs}, \emph{tamper evidence},
  jadwal audit triwulan.
\item
  \textbf{Penghapusan Data:} prosedur \emph{right to erasure}.
\end{itemize}

\begin{center}\rule{0.5\linewidth}{0.5pt}\end{center}

\subsection{F.12 Checklist Eksekusi (MVP 12
Minggu)}\label{f.12-checklist-eksekusi-mvp-12-minggu}

\begin{itemize}
\tightlist
\item[$\square$]
  Tim inti terbentuk \& pelatihan selesai
\item[$\square$]
  30 peserta \& 15 relawan aktif
\item[$\square$]
  3 stasiun berjalan (Relasi, Aktivitas, Ekonomi)
\item[$\square$]
  Ledger operasional \& dashboard dasar
\item[$\square$]
  SOP eskalasi aktif \& \emph{on‑call}
\item[$\square$]
  Review 4‑mingguan (\emph{gap closure}, insiden, biaya)
\end{itemize}

\begin{center}\rule{0.5\linewidth}{0.5pt}\end{center}

\subsection{Ucapan Terima Kasih}\label{ucapan-terima-kasih}

Terima kasih kepada seluruh kontributor konsep PSKVE,
Triune‑Intelligence, serta para praktisi dan peneliti yang menguji
gagasan ini di lapangan.

\subsection{Lisensi (opsional)}\label{lisensi-opsional}

Pilih lisensi terbuka/semi‑terbuka sesuai tujuan penyebarluasan artefak
\& instrumen.




\end{document}
