% Options for packages loaded elsewhere
% Options for packages loaded elsewhere
\PassOptionsToPackage{unicode}{hyperref}
\PassOptionsToPackage{hyphens}{url}
\PassOptionsToPackage{dvipsnames,svgnames,x11names}{xcolor}
%
\documentclass[
  letterpaper,
  DIV=11,
  numbers=noendperiod]{scrreprt}
\usepackage{xcolor}
\usepackage{amsmath,amssymb}
\setcounter{secnumdepth}{5}
\usepackage{iftex}
\ifPDFTeX
  \usepackage[T1]{fontenc}
  \usepackage[utf8]{inputenc}
  \usepackage{textcomp} % provide euro and other symbols
\else % if luatex or xetex
  \usepackage{unicode-math} % this also loads fontspec
  \defaultfontfeatures{Scale=MatchLowercase}
  \defaultfontfeatures[\rmfamily]{Ligatures=TeX,Scale=1}
\fi
\usepackage{lmodern}
\ifPDFTeX\else
  % xetex/luatex font selection
\fi
% Use upquote if available, for straight quotes in verbatim environments
\IfFileExists{upquote.sty}{\usepackage{upquote}}{}
\IfFileExists{microtype.sty}{% use microtype if available
  \usepackage[]{microtype}
  \UseMicrotypeSet[protrusion]{basicmath} % disable protrusion for tt fonts
}{}
\makeatletter
\@ifundefined{KOMAClassName}{% if non-KOMA class
  \IfFileExists{parskip.sty}{%
    \usepackage{parskip}
  }{% else
    \setlength{\parindent}{0pt}
    \setlength{\parskip}{6pt plus 2pt minus 1pt}}
}{% if KOMA class
  \KOMAoptions{parskip=half}}
\makeatother
% Make \paragraph and \subparagraph free-standing
\makeatletter
\ifx\paragraph\undefined\else
  \let\oldparagraph\paragraph
  \renewcommand{\paragraph}{
    \@ifstar
      \xxxParagraphStar
      \xxxParagraphNoStar
  }
  \newcommand{\xxxParagraphStar}[1]{\oldparagraph*{#1}\mbox{}}
  \newcommand{\xxxParagraphNoStar}[1]{\oldparagraph{#1}\mbox{}}
\fi
\ifx\subparagraph\undefined\else
  \let\oldsubparagraph\subparagraph
  \renewcommand{\subparagraph}{
    \@ifstar
      \xxxSubParagraphStar
      \xxxSubParagraphNoStar
  }
  \newcommand{\xxxSubParagraphStar}[1]{\oldsubparagraph*{#1}\mbox{}}
  \newcommand{\xxxSubParagraphNoStar}[1]{\oldsubparagraph{#1}\mbox{}}
\fi
\makeatother


\usepackage{longtable,booktabs,array}
\usepackage{calc} % for calculating minipage widths
% Correct order of tables after \paragraph or \subparagraph
\usepackage{etoolbox}
\makeatletter
\patchcmd\longtable{\par}{\if@noskipsec\mbox{}\fi\par}{}{}
\makeatother
% Allow footnotes in longtable head/foot
\IfFileExists{footnotehyper.sty}{\usepackage{footnotehyper}}{\usepackage{footnote}}
\makesavenoteenv{longtable}
\usepackage{graphicx}
\makeatletter
\newsavebox\pandoc@box
\newcommand*\pandocbounded[1]{% scales image to fit in text height/width
  \sbox\pandoc@box{#1}%
  \Gscale@div\@tempa{\textheight}{\dimexpr\ht\pandoc@box+\dp\pandoc@box\relax}%
  \Gscale@div\@tempb{\linewidth}{\wd\pandoc@box}%
  \ifdim\@tempb\p@<\@tempa\p@\let\@tempa\@tempb\fi% select the smaller of both
  \ifdim\@tempa\p@<\p@\scalebox{\@tempa}{\usebox\pandoc@box}%
  \else\usebox{\pandoc@box}%
  \fi%
}
% Set default figure placement to htbp
\def\fps@figure{htbp}
\makeatother


% definitions for citeproc citations
\NewDocumentCommand\citeproctext{}{}
\NewDocumentCommand\citeproc{mm}{%
  \begingroup\def\citeproctext{#2}\cite{#1}\endgroup}
\makeatletter
 % allow citations to break across lines
 \let\@cite@ofmt\@firstofone
 % avoid brackets around text for \cite:
 \def\@biblabel#1{}
 \def\@cite#1#2{{#1\if@tempswa , #2\fi}}
\makeatother
\newlength{\cslhangindent}
\setlength{\cslhangindent}{1.5em}
\newlength{\csllabelwidth}
\setlength{\csllabelwidth}{3em}
\newenvironment{CSLReferences}[2] % #1 hanging-indent, #2 entry-spacing
 {\begin{list}{}{%
  \setlength{\itemindent}{0pt}
  \setlength{\leftmargin}{0pt}
  \setlength{\parsep}{0pt}
  % turn on hanging indent if param 1 is 1
  \ifodd #1
   \setlength{\leftmargin}{\cslhangindent}
   \setlength{\itemindent}{-1\cslhangindent}
  \fi
  % set entry spacing
  \setlength{\itemsep}{#2\baselineskip}}}
 {\end{list}}
\usepackage{calc}
\newcommand{\CSLBlock}[1]{\hfill\break\parbox[t]{\linewidth}{\strut\ignorespaces#1\strut}}
\newcommand{\CSLLeftMargin}[1]{\parbox[t]{\csllabelwidth}{\strut#1\strut}}
\newcommand{\CSLRightInline}[1]{\parbox[t]{\linewidth - \csllabelwidth}{\strut#1\strut}}
\newcommand{\CSLIndent}[1]{\hspace{\cslhangindent}#1}



\setlength{\emergencystretch}{3em} % prevent overfull lines

\providecommand{\tightlist}{%
  \setlength{\itemsep}{0pt}\setlength{\parskip}{0pt}}



 


\KOMAoption{captions}{tableheading}
\makeatletter
\@ifpackageloaded{bookmark}{}{\usepackage{bookmark}}
\makeatother
\makeatletter
\@ifpackageloaded{caption}{}{\usepackage{caption}}
\AtBeginDocument{%
\ifdefined\contentsname
  \renewcommand*\contentsname{Table of contents}
\else
  \newcommand\contentsname{Table of contents}
\fi
\ifdefined\listfigurename
  \renewcommand*\listfigurename{List of Figures}
\else
  \newcommand\listfigurename{List of Figures}
\fi
\ifdefined\listtablename
  \renewcommand*\listtablename{List of Tables}
\else
  \newcommand\listtablename{List of Tables}
\fi
\ifdefined\figurename
  \renewcommand*\figurename{Figure}
\else
  \newcommand\figurename{Figure}
\fi
\ifdefined\tablename
  \renewcommand*\tablename{Table}
\else
  \newcommand\tablename{Table}
\fi
}
\@ifpackageloaded{float}{}{\usepackage{float}}
\floatstyle{ruled}
\@ifundefined{c@chapter}{\newfloat{codelisting}{h}{lop}}{\newfloat{codelisting}{h}{lop}[chapter]}
\floatname{codelisting}{Listing}
\newcommand*\listoflistings{\listof{codelisting}{List of Listings}}
\makeatother
\makeatletter
\makeatother
\makeatletter
\@ifpackageloaded{caption}{}{\usepackage{caption}}
\@ifpackageloaded{subcaption}{}{\usepackage{subcaption}}
\makeatother
\usepackage{bookmark}
\IfFileExists{xurl.sty}{\usepackage{xurl}}{} % add URL line breaks if available
\urlstyle{same}
\hypersetup{
  pdftitle={Triune-Intelligence Smart-Engineering},
  pdfauthor={Armein Z. R. Langi},
  colorlinks=true,
  linkcolor={blue},
  filecolor={Maroon},
  citecolor={Blue},
  urlcolor={Blue},
  pdfcreator={LaTeX via pandoc}}


\title{Triune-Intelligence Smart-Engineering}
\author{Armein Z. R. Langi}
\date{2025-10-09}
\begin{document}
\maketitle

\renewcommand*\contentsname{Table of contents}
{
\hypersetup{linkcolor=}
\setcounter{tocdepth}{2}
\tableofcontents
}

\bookmarksetup{startatroot}

\chapter*{Pengantar}\label{pengantar}
\addcontentsline{toc}{chapter}{Pengantar}

\markboth{Pengantar}{Pengantar}

Selamat datang di Kelompok Penelitian Triune Intelligence Smart
Engineering (TISE).

Di portal ini Anda dapat melihat daftar Topik tugas akhir, tesis, dan
disertasi TISE yang sedang dikerjakan maupun yang sedang ditawarkan.

Topik disusun berdasarkan tema atau topik besar, ynag kemudian diuraikan
ke dalam topik-topik yang lebih spesifik.

\bookmarksetup{startatroot}

\chapter{Pendahuluan}\label{pendahuluan}

This is a book created from markdown and executable code.

See Knuth (1984) for additional discussion of literate programming.

\bookmarksetup{startatroot}

\chapter{Topik Disertasi (S3)}\label{topik-disertasi-s3}

\section{Tema 1: Project and Financial
Engineering}\label{tema-1-project-and-financial-engineering}

\begin{enumerate}
\def\labelenumi{\arabic{enumi}.}
\item
  Judul: Model Value Index Untuk Rekayasa Produk Finansial dengan
  penerapannya pada Digital Loan.

  \begin{enumerate}
  \def\labelenumii{\arabic{enumii}.}
  \item
    Nomor Topik: D1.1
  \item
    Deskripsi Singkat:
  \item
    Mahasiwa: Sen Yung
  \item
    Status: Review Disertasi
  \end{enumerate}
\item
  Judul: Portofolio Project Optimisasi Model Menggunakan Pembiayaan
  Campuran

  \begin{enumerate}
  \def\labelenumii{\arabic{enumii}.}
  \item
    Nomor Topik: D1.2
  \item
    Deskripsi Singkat:
  \item
    Mahasiswa: Nisa Hanum
  \item
    Status: SK2
  \end{enumerate}
\end{enumerate}

\section{Tema 2: Quality of
Experience}\label{tema-2-quality-of-experience}

\begin{enumerate}
\def\labelenumi{\arabic{enumi}.}
\item
  Judul: Optimisasi QoE Layanan Video Streaming Berbasis Persepsi
  Pengguna

  \begin{enumerate}
  \def\labelenumii{\arabic{enumii}.}
  \item
    Nomor Topik: D2.1
  \item
    Deskripsi Singkat:
  \item
    Mahasiswa: Fahmi Candra Permana
  \item
    Status: SK2
  \end{enumerate}
\end{enumerate}

\section{Tema 3: Smart Living
Environment}\label{tema-3-smart-living-environment}

\begin{enumerate}
\def\labelenumi{\arabic{enumi}.}
\tightlist
\item
  Judul: GRACE index for Community Computing its Application to Elder
  Living Population

  \begin{enumerate}
  \def\labelenumii{\arabic{enumii}.}
  \tightlist
  \item
    Nomor Topik: D3.1
  \item
    Deskripsi Singkat:
  \item
    Mahasiswa: Tiur Gantini
  \item
    Status: Menuju SK3
  \end{enumerate}
\item
  Judul: Multistakeholder Recommendation System for Healthy Food

  \begin{enumerate}
  \def\labelenumii{\arabic{enumii}.}
  \tightlist
  \item
    Nomor Topik: D3.2
  \item
    Deskripsi Singkat:
  \item
    Mahasiswa: Laili Wahyunita
  \item
    Status: Menuju SK2
  \end{enumerate}
\end{enumerate}

\section{Tema 4: Smart Signal Processing for Human Therapy
Purposes}\label{tema-4-smart-signal-processing-for-human-therapy-purposes}

\begin{enumerate}
\def\labelenumi{\arabic{enumi}.}
\tightlist
\item
  Judul: Development an Adaptive EEG Model for Machine Learning based
  Attention Detection (Neurofeedback Therapy)

  \begin{enumerate}
  \def\labelenumii{\arabic{enumii}.}
  \tightlist
  \item
    Nomor Topik: D4.1
  \item
    Deskripsi Singkat:
  \item
    Mahasiswa: Teddy Marcus Zakaria (Kontak: 085795650857)
  \item
    Status: Menuju SK1
  \end{enumerate}
\end{enumerate}

\section{Tema 5: Smart Working Environment in Academic
Settings}\label{tema-5-smart-working-environment-in-academic-settings}

\begin{enumerate}
\def\labelenumi{\arabic{enumi}.}
\tightlist
\item
  Judul: Smart Campus Model

  \begin{enumerate}
  \def\labelenumii{\arabic{enumii}.}
  \tightlist
  \item
    Nomor Topik: D5.1
  \item
    Deskripsi Singkat:
  \item
    Mahasiswa: Radiant Victor Imbar
  \item
    Status: Selesai (Tahun 2024)
  \end{enumerate}
\item
  Judul: Framework Manajemen Cerdas Pengambilan Keputusan Menggunakan
  Pembelajaran Mesin dan Manajemen Pengetahuan untuk Mendukung
  Akreditasi Program Studi

  \begin{enumerate}
  \def\labelenumii{\arabic{enumii}.}
  \tightlist
  \item
    Nomor Topik: D5.2
  \item
    Deskripsi Singkat:
  \item
    Mahasiswa: Daniel Jahja Surjawan
  \item
    Status: Menuju SK1
  \end{enumerate}
\item
  Judul: Kerangka Kerja Manajemen Pengetahuan Kolaboratif Berbasis
  Rekayasa Cerdas untuk Pendidikan Tinggi Beriorientasi Nilai

  \begin{enumerate}
  \def\labelenumii{\arabic{enumii}.}
  \tightlist
  \item
    Nomor Topik: D5.3
  \item
    Deskripsi Singkat: Penelitian ini mengusulkan pengembangan Kerangka
    Sistem Manajemen Pengetahuan Kolaboratif Cerdas berbasis Rekayasa
    Cerdas (Collaborative Knowledge Management System-Smart
    Engineering/CKMS-SE) sebagai Smart Artefact. CKMS-SE dirancang
    dengan Core Engine (platform CKMS fundamental), PUDAL Engine
    (Perceive, Understand, Decision-Making and Planning, Act-Response,
    Learning-Evaluating) yang didukung oleh Kecerdasan Buatan
    (Artificial Intelligence/AI) untuk pembelajaran personal
    (Personalized Learning/PL) dinamis dan memfasilitasi pembelajaran
    kolaborasi (Collaborative Learning/CL) efektif, serta PSKVE Engine
    (Product, Service, Knowledge, Value, Environmental) untuk mengelola
    dan mengoptimalkan VCC dalam konteks VOE. Arsitektur sistem
    mengimplementasikan Triune-Intelligence Smart Engineering (TISE)
    yang menyinergikan kecerdasan manusia (Homocordium), kecerdasan
    buatan (Homologos), dan kecerdasan alamiah untuk memastikan
    penyelarasan nilai dalam sistem pembelajaran cerdas.
  \item
    Mahasiswa: Meliana Christianti Johan
  \item
    Status: Menuju SK1
  \end{enumerate}
\end{enumerate}

\bookmarksetup{startatroot}

\chapter{Topik Disertasi}\label{topik-disertasi}

\section{Tema 1: Project and Financial
Engineering}\label{tema-1-project-and-financial-engineering-1}

\begin{enumerate}
\def\labelenumi{\arabic{enumi}.}
\item
  Judul: Model Value Index Untuk Rekayasa Produk Finansial dengan
  penerapannya pada Digital Loan.

  \begin{enumerate}
  \def\labelenumii{\arabic{enumii}.}
  \item
    Nomor Topik: D1.1
  \item
    Deskripsi Singkat:
  \item
    Mahasiwa: Sen Yung
  \item
    Status: Review Disertasi
  \end{enumerate}
\item
  Judul: Portofolio Project Optimisasi Model Menggunakan Pembiayaan
  Campuran

  \begin{enumerate}
  \def\labelenumii{\arabic{enumii}.}
  \item
    Nomor Topik: D1.2
  \item
    Deskripsi Singkat:
  \item
    Mahasiswa: Nisa Hanum
  \item
    Status: SK2
  \end{enumerate}
\end{enumerate}

\section{Tema 2: Quality of
Experience}\label{tema-2-quality-of-experience-1}

\begin{enumerate}
\def\labelenumi{\arabic{enumi}.}
\item
  Judul: Optimisasi QoE Layanan Video Streaming Berbasis Persepsi
  Pengguna

  \begin{enumerate}
  \def\labelenumii{\arabic{enumii}.}
  \item
    Nomor Topik: D2.1
  \item
    Deskripsi Singkat:
  \item
    Mahasiswa: Fahmi Candra Permana
  \item
    Status: SK2
  \end{enumerate}
\end{enumerate}

\section{Tema 3: Smart Living
Environment}\label{tema-3-smart-living-environment-1}

\begin{enumerate}
\def\labelenumi{\arabic{enumi}.}
\tightlist
\item
  Judul: GRACE index for Community Computing its Application to Elder
  Living Population

  \begin{enumerate}
  \def\labelenumii{\arabic{enumii}.}
  \tightlist
  \item
    Nomor Topik: D3.1
  \item
    Deskripsi Singkat:
  \item
    Mahasiswa: Tiur Gantini
  \item
    Status: Menuju SK3
  \end{enumerate}
\item
  Judul: Multistakeholder Recommendation System for Healthy Food

  \begin{enumerate}
  \def\labelenumii{\arabic{enumii}.}
  \tightlist
  \item
    Nomor Topik: D3.2
  \item
    Deskripsi Singkat:
  \item
    Mahasiswa: Laili Wahyunita
  \item
    Status: Menuju SK2
  \end{enumerate}
\end{enumerate}

\section{Tema 4: Smart Signal Processing for Human Therapy
Purposes}\label{tema-4-smart-signal-processing-for-human-therapy-purposes-1}

\begin{enumerate}
\def\labelenumi{\arabic{enumi}.}
\tightlist
\item
  Judul: Development an Adaptive EEG Model for Machine Learning based
  Attention Detection (Neurofeedback Therapy)

  \begin{enumerate}
  \def\labelenumii{\arabic{enumii}.}
  \tightlist
  \item
    Nomor Topik: D4.1
  \item
    Deskripsi Singkat:
  \item
    Mahasiswa: Teddy Marcus Zakaria (Kontak: 085795650857)
  \item
    Status: Menuju SK1
  \end{enumerate}
\end{enumerate}

\section{Tema 5: Smart Working Environment in Academic
Settings}\label{tema-5-smart-working-environment-in-academic-settings-1}

\begin{enumerate}
\def\labelenumi{\arabic{enumi}.}
\tightlist
\item
  Judul: Smart Campus Model

  \begin{enumerate}
  \def\labelenumii{\arabic{enumii}.}
  \tightlist
  \item
    Nomor Topik: D5.1
  \item
    Deskripsi Singkat:
  \item
    Mahasiswa: Radiant Victor Imbar
  \item
    Status: Selesai (Tahun 2024)
  \end{enumerate}
\item
  Judul: Framework Manajemen Cerdas Pengambilan Keputusan Menggunakan
  Pembelajaran Mesin dan Manajemen Pengetahuan untuk Mendukung
  Akreditasi Program Studi

  \begin{enumerate}
  \def\labelenumii{\arabic{enumii}.}
  \tightlist
  \item
    Nomor Topik: D5.2
  \item
    Deskripsi Singkat:
  \item
    Mahasiswa: Daniel Jahja Surjawan
  \item
    Status: Menuju SK1
  \end{enumerate}
\item
  Judul: Kerangka Kerja Manajemen Pengetahuan Kolaboratif Berbasis
  Rekayasa Cerdas untuk Pendidikan Tinggi Beriorientasi Nilai

  \begin{enumerate}
  \def\labelenumii{\arabic{enumii}.}
  \tightlist
  \item
    Nomor Topik: D5.3
  \item
    Deskripsi Singkat:
  \item
    Mahasiswa: Meliana Christianti Johan
  \item
    Status: Menuju SK1
  \end{enumerate}
\end{enumerate}

\bookmarksetup{startatroot}

\chapter{Topik Disertasi}\label{topik-disertasi-1}

\section{Tema 1: Project and Financial
Engineering}\label{tema-1-project-and-financial-engineering-2}

\begin{enumerate}
\def\labelenumi{\arabic{enumi}.}
\item
  Judul: Model Value Index Untuk Rekayasa Produk Finansial dengan
  penerapannya pada Digital Loan.

  \begin{enumerate}
  \def\labelenumii{\arabic{enumii}.}
  \item
    Nomor Topik: D1.1
  \item
    Deskripsi Singkat:
  \item
    Mahasiwa: Sen Yung
  \item
    Status: Review Disertasi
  \end{enumerate}
\item
  Judul: Portofolio Project Optimisasi Model Menggunakan Pembiayaan
  Campuran

  \begin{enumerate}
  \def\labelenumii{\arabic{enumii}.}
  \item
    Nomor Topik: D1.2
  \item
    Deskripsi Singkat:
  \item
    Mahasiswa: Nisa Hanum
  \item
    Status: SK2
  \end{enumerate}
\end{enumerate}

\section{Tema 2: Quality of
Experience}\label{tema-2-quality-of-experience-2}

\begin{enumerate}
\def\labelenumi{\arabic{enumi}.}
\item
  Judul: Optimisasi QoE Layanan Video Streaming Berbasis Persepsi
  Pengguna

  \begin{enumerate}
  \def\labelenumii{\arabic{enumii}.}
  \item
    Nomor Topik: D2.1
  \item
    Deskripsi Singkat:
  \item
    Mahasiswa: Fahmi Candra Permana
  \item
    Status: SK2
  \end{enumerate}
\end{enumerate}

\section{Tema 3: Smart Living
Environment}\label{tema-3-smart-living-environment-2}

\begin{enumerate}
\def\labelenumi{\arabic{enumi}.}
\tightlist
\item
  Judul: GRACE index for Community Computing its Application to Elder
  Living Population

  \begin{enumerate}
  \def\labelenumii{\arabic{enumii}.}
  \tightlist
  \item
    Nomor Topik: D3.1
  \item
    Deskripsi Singkat:
  \item
    Mahasiswa: Tiur Gantini
  \item
    Status: Menuju SK3
  \end{enumerate}
\item
  Judul: Multistakeholder Recommendation System for Healthy Food

  \begin{enumerate}
  \def\labelenumii{\arabic{enumii}.}
  \tightlist
  \item
    Nomor Topik: D3.2
  \item
    Deskripsi Singkat:
  \item
    Mahasiswa: Laili Wahyunita
  \item
    Status: Menuju SK2
  \end{enumerate}
\end{enumerate}

\section{Tema 4: Smart Signal Processing for Human Therapy
Purposes}\label{tema-4-smart-signal-processing-for-human-therapy-purposes-2}

\begin{enumerate}
\def\labelenumi{\arabic{enumi}.}
\tightlist
\item
  Judul: Development an Adaptive EEG Model for Machine Learning based
  Attention Detection (Neurofeedback Therapy)

  \begin{enumerate}
  \def\labelenumii{\arabic{enumii}.}
  \tightlist
  \item
    Nomor Topik: D4.1
  \item
    Deskripsi Singkat:
  \item
    Mahasiswa: Teddy Marcus Zakaria (Kontak: 085795650857)
  \item
    Status: Menuju SK1
  \end{enumerate}
\end{enumerate}

\section{Tema 5: Smart Working Environment in Academic
Settings}\label{tema-5-smart-working-environment-in-academic-settings-2}

\begin{enumerate}
\def\labelenumi{\arabic{enumi}.}
\tightlist
\item
  Judul: Smart Campus Model

  \begin{enumerate}
  \def\labelenumii{\arabic{enumii}.}
  \tightlist
  \item
    Nomor Topik: D5.1
  \item
    Deskripsi Singkat:
  \item
    Mahasiswa: Radiant Victor Imbar
  \item
    Status: Selesai (Tahun 2024)
  \end{enumerate}
\item
  Judul: Framework Manajemen Cerdas Pengambilan Keputusan Menggunakan
  Pembelajaran Mesin dan Manajemen Pengetahuan untuk Mendukung
  Akreditasi Program Studi

  \begin{enumerate}
  \def\labelenumii{\arabic{enumii}.}
  \tightlist
  \item
    Nomor Topik: D5.2
  \item
    Deskripsi Singkat:
  \item
    Mahasiswa: Daniel Jahja Surjawan
  \item
    Status: Menuju SK1
  \end{enumerate}
\item
  Judul: Kerangka Kerja Manajemen Pengetahuan Kolaboratif Berbasis
  Rekayasa Cerdas untuk Pendidikan Tinggi Beriorientasi Nilai

  \begin{enumerate}
  \def\labelenumii{\arabic{enumii}.}
  \tightlist
  \item
    Nomor Topik: D5.3
  \item
    Deskripsi Singkat:
  \item
    Mahasiswa: Meliana Christianti Johan
  \item
    Status: Menuju SK1
  \end{enumerate}
\end{enumerate}

\bookmarksetup{startatroot}

\chapter{Topik Disertasi}\label{topik-disertasi-2}

\section{Tema 1: Project and Financial
Engineering}\label{tema-1-project-and-financial-engineering-3}

\begin{enumerate}
\def\labelenumi{\arabic{enumi}.}
\item
  Judul: Model Value Index Untuk Rekayasa Produk Finansial dengan
  penerapannya pada Digital Loan.

  \begin{enumerate}
  \def\labelenumii{\arabic{enumii}.}
  \item
    Nomor Topik: D1.1
  \item
    Deskripsi Singkat:
  \item
    Mahasiwa: Sen Yung
  \item
    Status: Review Disertasi
  \end{enumerate}
\item
  Judul: Portofolio Project Optimisasi Model Menggunakan Pembiayaan
  Campuran

  \begin{enumerate}
  \def\labelenumii{\arabic{enumii}.}
  \item
    Nomor Topik: D1.2
  \item
    Deskripsi Singkat:
  \item
    Mahasiswa: Nisa Hanum
  \item
    Status: SK2
  \end{enumerate}
\end{enumerate}

\section{Tema 2: Quality of
Experience}\label{tema-2-quality-of-experience-3}

\begin{enumerate}
\def\labelenumi{\arabic{enumi}.}
\item
  Judul: Optimisasi QoE Layanan Video Streaming Berbasis Persepsi
  Pengguna

  \begin{enumerate}
  \def\labelenumii{\arabic{enumii}.}
  \item
    Nomor Topik: D2.1
  \item
    Deskripsi Singkat:
  \item
    Mahasiswa: Fahmi Candra Permana
  \item
    Status: SK2
  \end{enumerate}
\end{enumerate}

\section{Tema 3: Smart Living
Environment}\label{tema-3-smart-living-environment-3}

\begin{enumerate}
\def\labelenumi{\arabic{enumi}.}
\tightlist
\item
  Judul: GRACE index for Community Computing its Application to Elder
  Living Population

  \begin{enumerate}
  \def\labelenumii{\arabic{enumii}.}
  \tightlist
  \item
    Nomor Topik: D3.1
  \item
    Deskripsi Singkat:
  \item
    Mahasiswa: Tiur Gantini
  \item
    Status: Menuju SK3
  \end{enumerate}
\item
  Judul: Multistakeholder Recommendation System for Healthy Food

  \begin{enumerate}
  \def\labelenumii{\arabic{enumii}.}
  \tightlist
  \item
    Nomor Topik: D3.2
  \item
    Deskripsi Singkat:
  \item
    Mahasiswa: Laili Wahyunita
  \item
    Status: Menuju SK2
  \end{enumerate}
\end{enumerate}

\section{Tema 4: Smart Signal Processing for Human Therapy
Purposes}\label{tema-4-smart-signal-processing-for-human-therapy-purposes-3}

\begin{enumerate}
\def\labelenumi{\arabic{enumi}.}
\tightlist
\item
  Judul: Development an Adaptive EEG Model for Machine Learning based
  Attention Detection (Neurofeedback Therapy)

  \begin{enumerate}
  \def\labelenumii{\arabic{enumii}.}
  \tightlist
  \item
    Nomor Topik: D4.1
  \item
    Deskripsi Singkat:
  \item
    Mahasiswa: Teddy Marcus Zakaria (Kontak: 085795650857)
  \item
    Status: Menuju SK1
  \end{enumerate}
\end{enumerate}

\section{Tema 5: Smart Working Environment in Academic
Settings}\label{tema-5-smart-working-environment-in-academic-settings-3}

\begin{enumerate}
\def\labelenumi{\arabic{enumi}.}
\tightlist
\item
  Judul: Smart Campus Model

  \begin{enumerate}
  \def\labelenumii{\arabic{enumii}.}
  \tightlist
  \item
    Nomor Topik: D5.1
  \item
    Deskripsi Singkat:
  \item
    Mahasiswa: Radiant Victor Imbar
  \item
    Status: Selesai (Tahun 2024)
  \end{enumerate}
\item
  Judul: Framework Manajemen Cerdas Pengambilan Keputusan Menggunakan
  Pembelajaran Mesin dan Manajemen Pengetahuan untuk Mendukung
  Akreditasi Program Studi

  \begin{enumerate}
  \def\labelenumii{\arabic{enumii}.}
  \tightlist
  \item
    Nomor Topik: D5.2
  \item
    Deskripsi Singkat:
  \item
    Mahasiswa: Daniel Jahja Surjawan
  \item
    Status: Menuju SK1
  \end{enumerate}
\item
  Judul: Kerangka Kerja Manajemen Pengetahuan Kolaboratif Berbasis
  Rekayasa Cerdas untuk Pendidikan Tinggi Beriorientasi Nilai

  \begin{enumerate}
  \def\labelenumii{\arabic{enumii}.}
  \tightlist
  \item
    Nomor Topik: D5.3
  \item
    Deskripsi Singkat:
  \item
    Mahasiswa: Meliana Christianti Johan
  \item
    Status: Menuju SK1
  \end{enumerate}
\end{enumerate}

\bookmarksetup{startatroot}

\chapter{Topik Tesis}\label{topik-tesis}

Berikut adalah daftar topik penelitian Tesis (S2) dan Tugas Akhir (S1)
yang diturunkan dari riset disertasi S3 mengenai \emph{Multi-Stakeholder
Food Recommendation System}, beserta deskripsi singkatnya.

\section{\texorpdfstring{\textbf{Daftar Topik Tesis (S2) MSRS -
Kontribusi pada Layer
Sistem}}{Daftar Topik Tesis (S2) MSRS - Kontribusi pada Layer Sistem}}\label{daftar-topik-tesis-s2-msrs---kontribusi-pada-layer-sistem}

\begin{enumerate}
\def\labelenumi{\arabic{enumi}.}
\tightlist
\item
  \textbf{Arsitektur Sistem MSRS Berbasis Microservices untuk Integrasi
  Pemangku Kepentingan yang Skalabel}

  \begin{itemize}
  \tightlist
  \item
    \textbf{Deskripsi Singkat:} Merancang cetak biru (arsitektur) sistem
    MSRS secara keseluruhan menggunakan pendekatan \emph{microservices}.
    Penelitian ini berfokus pada cara mengintegrasikan data dan
    kebutuhan dari berbagai pihak (konsumen, produsen, ahli gizi) secara
    fleksibel dan andal, serta mendefinisikan API sebagai ``jembatan''
    komunikasi antar komponen.
  \end{itemize}
\item
  \textbf{Pengembangan Ontologi dan Kerangka Tata Kelola Data untuk
  Ekosistem Pangan Sehat}

  \begin{itemize}
  \tightlist
  \item
    \textbf{Deskripsi Singkat:} Membangun model pengetahuan formal
    (ontologi) yang menjadi ``kamus bersama'' untuk merepresentasikan
    seluruh entitas dan hubungan dalam sistem pangan (misal: nutrisi,
    kapasitas produksi, preferensi diet). Penelitian ini juga merancang
    aturan tata kelola untuk memastikan data yang digunakan berkualitas,
    aman, dan privat.
  \end{itemize}
\end{enumerate}

\section{Topik Tesis Magister (S2)
VALORIZE}\label{topik-tesis-magister-s2-valorize}

Berikut adalah daftar topik penelitian Tesis (S2) dan Tugas Akhir (S1)
yang diturunkan dari riset disertasi S3 mengenai \emph{VALORIZE System},
beserta deskripsi singkatnya.

\begin{itemize}
\tightlist
\item
  \textbf{S2.1: Desain dan Arsitektur Sistem Pasar Pengetahuan PSKVE
  Berbasis \emph{Microservices}}

  \begin{itemize}
  \tightlist
  \item
    \textbf{Deskripsi Singkat:} Merancang arsitektur sistem yang lengkap
    dan dapat diskalakan untuk Pasar Pengetahuan TISE-Valorize dengan
    menerapkan prinsip-prinsip \emph{microservices} untuk memastikan
    ketahanan dan kemudahan pemeliharaan sistem.{[}1, 2, 3, 4{]}
  \end{itemize}
\item
  \textbf{S2.2: Kerangka Kerja \emph{Model-Based Systems Engineering}
  (MBSE) untuk Alur Kerja Kognitif TISE-VALORIZE}

  \begin{itemize}
  \tightlist
  \item
    \textbf{Deskripsi Singkat:} Menerapkan metodologi rekayasa sistem
    formal (MBSE) menggunakan SysML untuk menciptakan model yang presisi
    dan dapat dieksekusi dari seluruh alur kerja pedagogis
    TISE-VALORIZE, guna memverifikasi konsistensi logisnya.{[}1, 5, 6{]}
  \end{itemize}
\item
  \textbf{S2.3: Arsitektur \emph{Data Warehousing} dan ETL untuk
  Platform Analitik Pembelajaran VALORIZE}

  \begin{itemize}
  \tightlist
  \item
    \textbf{Deskripsi Singkat:} Merancang arsitektur data
    \emph{back-end} dan proses \emph{Extract, Transform, Load} (ETL)
    yang diperlukan untuk secara sistematis menangkap, mengubah, dan
    menganalisis data kaya yang dihasilkan oleh mahasiswa di dalam
    ekosistem VALORIZE.{[}7, 8{]}
  \end{itemize}
\end{itemize}

Tentu, berikut adalah daftar topik riset turunan dari disertasi S3
``GRACE System'' beserta deskripsi singkatnya, dikelompokkan untuk
jenjang Tesis Magister (S2) dan Tugas Akhir Sarjana (S1).

\begin{center}\rule{0.5\linewidth}{0.5pt}\end{center}

\section{\texorpdfstring{\textbf{Daftar Topik Tesis (S2) GRACE - Layer
Sistem}}{Daftar Topik Tesis (S2) GRACE - Layer Sistem}}\label{daftar-topik-tesis-s2-grace---layer-sistem}

\begin{enumerate}
\def\labelenumi{\arabic{enumi}.}
\item
  \textbf{Arsitektur Sosio-Teknis untuk ``Wisdom as a Service'' pada
  Platform C-GRACE:} Merancang arsitektur sistem yang mengintegrasikan
  platform teknis dengan model sosial (reputasi, insentif) untuk
  memfasilitasi lansia berbagi keahlian dan kearifan mereka secara
  efektif dan adil.
\item
  \textbf{Desain Sistem Interoperabilitas Data Kesehatan pada O-GRACE:}
  Merancang arsitektur sistem yang aman dan terstandar untuk
  mengumpulkan dan mengelola data dari berbagai perangkat kesehatan
  (IoT) pihak ketiga, guna mendukung rekomendasi kesehatan yang
  dipersonalisasi bagi lansia.
\item
  \textbf{Sistem Manajemen Identitas dan Privasi Terdistribusi untuk
  Ekosistem GRACE:} Merancang sebuah sistem yang memungkinkan lansia
  memiliki kontrol penuh atas data pribadi mereka di seluruh subsistem
  GRACE (I, C, O, F) menggunakan prinsip-prinsip \emph{Self-Sovereign
  Identity} (SSI).
\item
  \textbf{Model Tata Kelola Komunitas Adaptif untuk C-GRACE Berbasis
  AI:} Mengembangkan arsitektur sistem tata kelola (governance) yang
  menggunakan AI untuk menganalisis dinamika komunitas dan
  merekomendasikan aturan atau intervensi guna menjaga lingkungan yang
  positif dan produktif.
\end{enumerate}

\begin{center}\rule{0.5\linewidth}{0.5pt}\end{center}

Berikut ini adalah daftar topik riset S2 dan S1 yang diturunkan dari
disertasi S3 ``Model Adaptif EEG Untuk Deteksi Perhatian Berbasis
Machine Learning'', beserta deskripsi singkatnya.

\section{\texorpdfstring{\textbf{Topik Tesis (S2 -
Magister)}}{Topik Tesis (S2 - Magister)}}\label{topik-tesis-s2---magister}

\begin{itemize}
\tightlist
\item
  \textbf{Topik 1: Rancang Bangun Sistem Neurofeedback Adaptif
  \emph{Real-time} untuk Peningkatan Fokus Belajar Menggunakan Sinyal
  EEG.}

  \begin{itemize}
  \tightlist
  \item
    \textbf{Deskripsi Singkat:} Mengembangkan arsitektur sistem yang
    mengintegrasikan model AI pendeteksi perhatian dengan perangkat EEG
    dan antarmuka pengguna untuk memberikan umpan balik \emph{real-time}
    guna melatih fokus.
  \end{itemize}
\end{itemize}

\bookmarksetup{startatroot}

\chapter{Topik Tugas Akhir}\label{topik-tugas-akhir}

\section{\texorpdfstring{\textbf{Daftar Topik Tugas Akhir (S1) -
Kontribusi pada Layer Aplikasi
MSRS}}{Daftar Topik Tugas Akhir (S1) - Kontribusi pada Layer Aplikasi MSRS}}\label{daftar-topik-tugas-akhir-s1---kontribusi-pada-layer-aplikasi-msrs}

\begin{enumerate}
\def\labelenumi{\arabic{enumi}.}
\tightlist
\item
  \textbf{Pengembangan Aplikasi Mobile MSRS untuk Peningkatan Gizi dan
  Dukungan UMKM Lokal (Studi Kasus: Lingkungan Kampus)}

  \begin{itemize}
  \tightlist
  \item
    \textbf{Deskripsi Singkat:} Merancang dan membangun prototipe
    aplikasi \emph{mobile} yang memberikan rekomendasi makanan sehat
    dari kantin atau UMKM di sekitar kampus. Aplikasi ini akan
    divalidasi langsung oleh pengguna (mahasiswa) untuk menguji
    kemudahan penggunaan dan dampaknya terhadap pilihan makanan.
  \end{itemize}
\item
  \textbf{Rancang Bangun Dashboard Visualisasi Data untuk Pemberdayaan
  Produsen Pangan (UMKM/Petani)}

  \begin{itemize}
  \tightlist
  \item
    \textbf{Deskripsi Singkat:} Membuat sebuah aplikasi \emph{dashboard}
    (berbasis web atau \emph{mobile}) yang ditujukan khusus untuk
    produsen. \emph{Dashboard} ini akan menampilkan visualisasi data
    tren permintaan pasar, preferensi konsumen, dan umpan balik,
    sehingga membantu produsen mengambil keputusan bisnis yang lebih
    baik.
  \end{itemize}
\item
  \textbf{Sistem Gamifikasi pada Aplikasi MSRS untuk Mendorong Kebiasaan
  Makan Sehat di Kalangan Remaja}

  \begin{itemize}
  \tightlist
  \item
    \textbf{Deskripsi Singkat:} Mengimplementasikan fitur-fitur
    gamifikasi (seperti poin, lencana, dan tantangan) ke dalam aplikasi
    MSRS. Tujuan utamanya adalah untuk meningkatkan keterlibatan
    (\emph{engagement}) dan motivasi pengguna, khususnya remaja, dalam
    memilih dan mempertahankan pola makan yang sehat.
  \end{itemize}
\item
  \textbf{Analisis Sentimen dan Klasifikasi Umpan Balik Pengguna pada
  Produk UMKM Menggunakan Natural Language Processing (NLP)}

  \begin{itemize}
  \tightlist
  \item
    \textbf{Deskripsi Singkat:} Membangun model NLP untuk secara
    otomatis menganalisis dan mengklasifikasikan ulasan atau umpan balik
    yang diberikan pengguna terhadap produk makanan dari UMKM. Hasil
    analisis ini dapat memberikan wawasan kualitatif yang berharga bagi
    produsen untuk meningkatkan kualitas produk mereka.
  \end{itemize}
\end{enumerate}

\section{Topik Tugas Akhir Sarjana (S1)
VALORIZE}\label{topik-tugas-akhir-sarjana-s1-valorize}

\begin{itemize}
\tightlist
\item
  \textbf{S1.1: Aplikasi untuk Visualisasi dan Manajemen
  \emph{Real-Time} Portofolio PSKVE Mahasiswa}

  \begin{itemize}
  \tightlist
  \item
    \textbf{Deskripsi Singkat:} Merancang, mengembangkan, dan
    mengevaluasi sebuah aplikasi dasbor berbasis web yang berfungsi
    sebagai antarmuka utama bagi mahasiswa dan dosen untuk berinteraksi
    dan memvisualisasikan portofolio nilai PSKVE mereka.{[}1, 2, 9{]}
  \end{itemize}
\item
  \textbf{S1.2: Pengembangan dan Validasi Alat Penilaian dan Rubrik
  untuk Peta Kognitif Formal}

  \begin{itemize}
  \tightlist
  \item
    \textbf{Deskripsi Singkat:} Mengembangkan sebuah rubrik penilaian
    yang terperinci dan alat pendukungnya untuk menilai kualitas peta
    kognitif formal, serta memvalidasi keandalannya secara ilmiah
    menggunakan statistik reliabilitas antar-penilai.{[}2, 10, 11{]}
  \end{itemize}
\item
  \textbf{S1.3: Studi Kasus Penerapan Kerangka Kerja TISE PUDAL/TI pada
  Pengurangan Sampah Makanan}

  \begin{itemize}
  \tightlist
  \item
    \textbf{Deskripsi Singkat:} Melakukan studi kasus mendalam dengan
    menerapkan alur kerja kognitif TISE (PUDAL/\emph{Triune
    Intelligence}) secara penuh untuk menganalisis dan mengusulkan
    solusi pada sub-masalah spesifik dalam krisis keberlanjutan pangan
    di Indonesia.{[}1, 12{]}
  \end{itemize}
\item
  \textbf{S1.4: Evaluasi Usabilitas dan Efikasi Lingkungan Belajar
  TISE-VALORIZE}

  \begin{itemize}
  \tightlist
  \item
    \textbf{Deskripsi Singkat:} Melakukan studi penelitian formal dengan
    metode campuran untuk mengukur usabilitas kuantitatif dari alat-alat
    perangkat lunak dan mengeksplorasi persepsi kualitatif mahasiswa
    mengenai efektivitas pedagogis dari lingkungan belajar
    TISE-VALORIZE.{[}13, 14{]}
  \end{itemize}
\end{itemize}

\section{\texorpdfstring{\textbf{Daftar Topik Tugas Akhir (S1) - Layer
Aplikasi
GRACE}}{Daftar Topik Tugas Akhir (S1) - Layer Aplikasi GRACE}}\label{daftar-topik-tugas-akhir-s1---layer-aplikasi-grace}

\begin{enumerate}
\def\labelenumi{\arabic{enumi}.}
\item
  \textbf{Perancangan dan Evaluasi Antarmuka ``GRACE Life Map'' untuk
  Refleksi Diri Lansia:} Merancang dan menguji usabilitas sebuah
  aplikasi visualisasi narasi kehidupan yang intuitif dan menarik bagi
  lansia untuk membantu mereka merenungkan dan menceritakan kisah
  hidupnya.
\item
  \textbf{Implementasi Modul Gamifikasi untuk Mendorong Kontribusi
  Lansia di C-GRACE:} Mengembangkan dan menguji aplikasi modul
  gamifikasi (poin, lencana, papan peringkat) yang dirancang khusus
  untuk memotivasi partisipasi lansia dalam berbagai kegiatan komunitas.
\item
  \textbf{Aplikasi ``Perekam Cerita Keluarga'' pada F-GRACE untuk
  Komunikasi Antargenerasi:} Membangun aplikasi seluler yang memandu
  lansia dan anggota keluarga yang lebih muda untuk merekam cerita dan
  kenangan bersama secara interaktif.
\item
  \textbf{Prototipe Chatbot ``GRACE Health Companion'' untuk Dukungan
  Kesehatan Harian:} Mengembangkan aplikasi chatbot sederhana yang dapat
  memberikan pengingat minum obat, tips gaya hidup sehat, dan informasi
  kesehatan dasar kepada lansia melalui antarmuka percakapan.
\item
  \textbf{Pengembangan Aplikasi ``Direktori Aktivitas Sosial''
  Terpersonalisasi untuk Lansia:} Membuat aplikasi yang memberikan
  rekomendasi kegiatan sosial atau komunitas di sekitar lokasi lansia,
  sesuai dengan minat dan kemampuan fisik mereka.
\end{enumerate}

\section{\texorpdfstring{\textbf{Topik Tugas Akhir (S1 - Sarjana)
Neurofeedback}}{Topik Tugas Akhir (S1 - Sarjana) Neurofeedback}}\label{topik-tugas-akhir-s1---sarjana-neurofeedback}

\begin{itemize}
\tightlist
\item
  \textbf{Topik 1: Penerapan dan Analisis Sistem Deteksi Perhatian EEG
  untuk Mengukur Keterlibatan Mahasiswa pada Platform Pembelajaran
  Daring.}

  \begin{itemize}
  \tightlist
  \item
    \textbf{Deskripsi Singkat:} Menguji dan memvalidasi sistem deteksi
    perhatian dalam lingkungan \emph{e-learning} untuk melihat korelasi
    antara tingkat perhatian yang diukur EEG dengan hasil belajar
    mahasiswa.
  \end{itemize}
\item
  \textbf{Topik 2: Studi Komparatif Efektivitas Umpan Balik Visual
  vs.~Auditori dalam Sistem Neurofeedback EEG untuk Latihan Meditasi dan
  Relaksasi.}

  \begin{itemize}
  \tightlist
  \item
    \textbf{Deskripsi Singkat:} Membandingkan dua jenis umpan balik
    (visual dan suara) pada sistem \emph{neurofeedback} untuk menentukan
    mana yang lebih efektif dalam membantu pengguna mencapai kondisi
    relaksasi.
  \end{itemize}
\end{itemize}

\bookmarksetup{startatroot}

\chapter{Lankah Selajutnya}\label{lankah-selajutnya}

\bookmarksetup{startatroot}

\chapter*{References}\label{references}
\addcontentsline{toc}{chapter}{References}

\markboth{References}{References}

\phantomsection\label{refs}
\begin{CSLReferences}{1}{0}
\bibitem[\citeproctext]{ref-knuth84}
Knuth, Donald E. 1984. {``Literate Programming.''} \emph{Comput. J.} 27
(2): 97--111. \url{https://doi.org/10.1093/comjnl/27.2.97}.

\end{CSLReferences}




\end{document}
